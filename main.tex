\documentclass[a4paper, fontsize=12pt, parskip=full, toc=bibliographynumbered]{scrreprt}
\setcounter{tocdepth}{3}
\setcounter{secnumdepth}{3}
\usepackage[T1]{fontenc}
\usepackage[utf8]{inputenc}
\usepackage{mathptmx}
\usepackage{tgheros}
\usepackage{courier}
\usepackage{microtype}
\usepackage[onehalfspacing]{setspace}
\usepackage[defaultlines=2, all]{nowidow}
\usepackage[left=4cm, right=2cm, top=2.5cm, bottom=2.5cm]{geometry}
\usepackage[ngerman]{babel}
\usepackage[usenames,dvipsnames]{color}
\usepackage{graphicx}
\usepackage{tabu}
\usepackage{listings}
\usepackage{float}
\usepackage[footnote]{acronym}
\usepackage[ngerman]{cleveref}
\usepackage{eurosym}
\usepackage{amsmath}
\usepackage{amssymb}
\usepackage{thmbox}
\usepackage{shadethm}
\usepackage{booktabs}
\usepackage{hyperref}
\makeatletter
\g@addto@macro{\maketitle}{\def\author#1{\def\@author{#1}}}
\newcommand*{\extendsectlevel}[1]{%
  \expandafter\newcommand\expandafter*\csname saved@#1\endcsname{}%
  \expandafter\let\csname saved@#1\expandafter\endcsname\csname #1\endcsname
  \expandafter\renewcommand\expandafter*\csname #1\endcsname{%
    \expandafter\let\csname author@#1\endcsname\@author
    \@ifstar
      {\csname star@#1\endcsname}%
      {\@dblarg{\csname opt@#1\endcsname}}%
  }%
  \expandafter\newcommand\expandafter*\csname star@#1\endcsname[1]{%
    \csname saved@#1\endcsname*{##1%
      \expandafter\ifx\csname author@#1\endcsname\@empty\else
        \hfill\linebreak{\normalsize
          \textmd{\textit{\csname author@#1\endcsname}}}%
      \fi
    }%
  }%
  \expandafter\newcommand\expandafter*\csname opt@#1\endcsname[2][]{%
    \csname saved@#1\endcsname[{##1%
      \expandafter\ifx\csname author@#1\endcsname\@empty\else
        \enskip\textmd{\textit{(\csname author@#1\endcsname)}}%
      \fi
    }]{##2%
      \expandafter\ifx\csname author@#1\endcsname\@empty\else
        \hfill\linebreak{\normalsize
          \textmd{\textit{\csname author@#1\endcsname}}}%
      \fi
    }%
  }%
}
\extendsectlevel{chapter}
\makeatother
\begin{document}
\newcommand{\headerrule}{\tabucline -}
\newcommand{\abbildung}[2]{\begin{figure}\centering
    \fbox{\includegraphics[width=0.8\textwidth]{#1}}\caption{#2}
    \label{fig:#1}\end{figure}}
\setkomafont{chapterentry}{\bfseries}

\author{}
\begin{titlepage}
  \begin{center}
    \textsc{\large Fachhochschule der Wirtschaft\\FHDW}\\[1em]
    \textsc{\large Bergisch Gladbach}\\[2em]
    \textsc{Schriftliche Ausarbeitung}\\[6em]
    {\LARGE Kryptographie II}\\[25em]
    \begin{tabu} to 0.8\textwidth {X[l] X[r]}
      \emph{Autoren:}\linebreak
      An-Nam \textsc{Pham},\linebreak
      Leonie \textsc{Schiburr},\linebreak
      Patrick \textsc{Künzl},\linebreak
      Vasilij \textsc{Schneidermann}
      &
      \emph{Prüfer:}\linebreak
      Ralf \textsc{Schumann}
    \end{tabu}
    \vfill
    \emph{Abgabetermin:}\\
    \today
  \end{center}
\end{titlepage}

% Akronyme

\newacro{PKI}{Public Key Infrastructure}
\newacro{RSA}{Rivest, Shamir, Adleman}
\newacro{MAC}{Message Authentication Code}
\newacro{HMAC}{Hash-based Message Authentication Code}
\newacro{MD5}{Message Digest 5}
\newacro{SHA1}{Secure Hash Algorithm 1}
\newacro{SHA2}{Secure Hash Algorithm 2}
\newacro{SHA3}{Secure Hash Algorithm 3}

\pagenumbering{roman}
\tableofcontents
\listoffigures
\listoftables
\lstlistoflistings
\clearpage
\pagenumbering{arabic}
\setcounter{page}{1}

\author{Vasilij Schneidermann}
\chapter{Einleitung}

Asymmetrische Kryptographie ist super!

%!TEX root = main.tex
\author{Autor: An-Nam Pham}
\chapter{Mathematische Grundlagen}

\newshadetheorem{Einweg}{Definition}
\newshadetheorem{Euler}{Definition}
\newshadetheorem{Faktorisierung}{Beweis}
%https://de.sharelatex.com/learn/Theorems_and_proofs

Alle Kryptosysteme haben eine mathematische Grundlage. Das asymmetrische Kryptosystem ist hier keine Ausnahme.
Deswegen soll in diesem Kapitel auf die Grundlagen der Mathematik hinter den asymmetrischen Kryptosystemen und vor allem dem RSA-Kryptosystem eingegangen werden.
%%%%%%%%%%%%%%%%%%%%%%%%%%%%%%%%%%%%%%%%%%%%%%%%%%%%%%%%%%%%%%%%%%%%%%%%%%%%%%%%%%%%%%
%%%%%%%%%%%%%%%%%%%%%%%%%%%%%%%%%%%%%%%%%%%%%%%%%%%%%%%%%%%%%%%%%%%%%%%%%%%%%%%%%%%%%%
%%%%%%%%%%%%%%%%%%%%%%%%%%%%%%%%%%%%%%%%%%%%%%%%%%%%%%%%%%%%%%%%%%%%%%%%%%%%%%%%%%%%%%
%%%%%%%%%%%%%%%%%%%%%%%%%%%%%%%%%%%%%%%%%%%%%%%%%%%%%%%%%%%%%%%%%%%%%%%%%%%%%%%%%%%%%%
%%%%%%%%%%%%%%%%%%%%%%%%%%%%%%%%%%%%%%%%%%%%%%%%%%%%%%%%%%%%%%%%%%%%%%%%%%%%%%%%%%%%%%
\section{Restklassenringe}
\label{sec:Restklassenringe}
% Restklassenringe bilden die Basis für diverse Kryptosysteme (z.B. RSA).
Ein Restklassenring ist eine algebraische Struktur mit Restklassen\footnote{Die Restklasse einer Zahl $a~mod(n)$ ist die Menge aller Zahlen, bei der die Division von $a$ durch $n$ denselben Rest haben wir $a$.} als Menge.
In einem Restklassenring lässt es sich so rechnen, wie man z.B. mit Uhrzeiten rechnen würde.\\
\\
\textbf{Beispiel}\\
Man kann Uhrzeiten addieren. Zum Beispiel:
\begin{equation*}
13~Uhr + 2~Stunden = 15~Uhr
\end{equation*}
Das Ergebnis kann aber nicht über 24 Uhr liegen:
\begin{equation*}
22~Uhr + 5~Stunden = 27~Uhr
\end{equation*}
Wird allerdings definiert, dass alle Werte Elemente der Restklasse $24\mathbb{Z}$ sind, existieren genau 24 Elemente. Nämlich die Zahlen 0 bis 23. Wenn man noch Addition und Multiplikation definiert, erhält man den Restklassenring $\mathbb{Z}/24\mathbb{Z}$, oder anders geschrieben: $\mathbb{Z}_{24}$.
In diesem Restklassenring sieht die vorherige Rechnung nun wie folgt aus:
\begin{equation*}
22~Uhr + 5~Stunden = 3~Uhr\quad in~\mathbb{Z}_{24}
\end{equation*}
oder mathematisch:
\begin{equation*}
22+5 \equiv 3~(mod~24)
\end{equation*}
In RSA wird zum Teil in einem Restklassenring gearbeitet, um die Voraussetzung für eine Einswegsfunktion zu schaffen (siehe Nachtrag auf Seite \pageref{sec:Nachtrag})\footnote{vgl. Rehn (2012)}.
%%%%%%%%%%%%%%%%%%%%%%%%%%%%%%%%%%%%%%%%%%%%%%%%%%%%%%%%%%%%%%%%%%%%%%%%%%%%%%%%%%%%%%
%%%%%%%%%%%%%%%%%%%%%%%%%%%%%%%%%%%%%%%%%%%%%%%%%%%%%%%%%%%%%%%%%%%%%%%%%%%%%%%%%%%%%%
%%%%%%%%%%%%%%%%%%%%%%%%%%%%%%%%%%%%%%%%%%%%%%%%%%%%%%%%%%%%%%%%%%%%%%%%%%%%%%%%%%%%%%
%%%%%%%%%%%%%%%%%%%%%%%%%%%%%%%%%%%%%%%%%%%%%%%%%%%%%%%%%%%%%%%%%%%%%%%%%%%%%%%%%%%%%%
%%%%%%%%%%%%%%%%%%%%%%%%%%%%%%%%%%%%%%%%%%%%%%%%%%%%%%%%%%%%%%%%%%%%%%%%%%%%%%%%%%%%%%
\section{Einwegfunktionen}
\label{sec:Einwegfunktionen}
Eine Einwegfunktion \textit{(engl. one-way-function)} ist die Grundlage aller asymmetrischen Kryptosysteme.

\begin{Einweg}[Einwegfunktion]
\label{Def_Einweg}
Eine Einwegfunktion ist gegeben,\\
wenn für eine injektive Funktion \(f:X\rightarrow Y\) folgendes gilt:
\begin{itemize}
\item Es gibt ein effizientes Verfahren zur Bestimmung von $y=f(x)~~\forall~~x \in X $
\item Es gibt kein effizientes Verfahren zu Bestimmung von $x=f^{-1}(y)~~\forall~~y \in f(X)$
\end{itemize}
\end{Einweg} 
Auf gut Deutsch sagt die Definition \ref{Def_Einweg}, dass eine Einwegfunktion einfach zu berechnen ist, die Umkehrung aber praktisch nicht durchführbar ist.

% Für die Präsi kann man auch noch folgende Beispiele bringen:
% Sich ein Bein brechen, Zahnpasta aus der Tube drücken, Brief zerreißen

\begin{description}
\item[Beispiel 1]\hfill \\
$f_{Telefon}:Name\rightarrow Telefonnummer$\\
Ein einfaches Beispiel für eine Einwegfunktion ist ein klassisches Telefonbuch. Hier kann man schnell zu jedem eingetragenen Namen eine Telefonnummer finden. Will man allerdings zu einer gegebenen Telefonnummer einen Namen finden, ist dies sehr aufwendig.
\item[Beispiel 2]\hfill \\
$f_{Stradivari}:Konstruktionsanweisung\rightarrow Stradivari$\\
Obwohl verschiedene Stradivari mehrfach analysiert wurden, lässt sich bis heute keine Kopie einer Violine mit einem ähnlichen Klanggefühl herstellen.
\end{description}
Das große Dilemma der Definition \ref{Def_Einweg} ist die erforderliche Tatsache, dass eine Einwegfunktion so schwierig umzukehren ist, dass es praktisch nicht umsetzbar ist.\\
Wird eine Einwegfunktion in einem Kryptosystem eingesetzt, kann die verschlüsselte Nachricht zwar nicht von Unbefugten geknackt werden, aber auch der rechtmäßige Empfänger wird nicht in der Lage sein, die Nachricht zu entschlüsseln.
\\
Es wird also eine Einwegfunktion mit Falltür \textit{(engl. trap-door one-way-funktion)} benötigt.

\begin{Einweg}[Einwegfunktion mit Falltür]
\label{Def_Einweg_Backdoor}
Eine Einwegfunktion mit Falltür ist gegeben,\\
wenn für eine injektive Funktion \(f:X\rightarrow Y\) folgendes gilt:
\begin{itemize}
\item Es gibt ein effizientes Verfahren $A$ zur Bestimmung von $y=f(x)~~\forall~~x \in X $
\item Es gibt ein effizientes Verfahren $B$ zu Bestimmung von $x=f^{-1}(y)~~\forall~~y \in f(X)$
\item Die Herleitung von $B$ aus $A$ ist  ohne eine geheimzuhaltene\\Falltürinformation sehr schwer bzw. nicht möglich.
\end{itemize}
\end{Einweg}
\hfill
\begin{description}
\item[Beispiel für eine Einwegfunktion mit Falltür]\hfill \\
Wenn man eine Tür ins Schloss fallen lässt, lässt sich diese Tür nur noch mit einem passenden Schlüssel öffnen.
\end{description}
Für das RSA-Kryptosystem wird eine solche Einwegfunktion mit Falltür verwendet. Die Einwegfunktion basiert auf dem Faktorisierungsproblem und die Falltür ist die Bestimmung des privaten Schlüssels mit Hilfe des öffentlichen Schlüssels. Die geheimzuhaltende Falltürinformation sind die beiden Primfaktoren der Zahl, die in beiden Schlüsseln vorhanden sind\footnote{vgl. MathePrisma (2001)}.
%%%%%%%%%%%%%%%%%%%%%%%%%%%%%%%%%%%%%%%%%%%%%%%%%%%%%%%%%%%%%%%%%%%%%%%%%%%%%%%%%%%%%%
%%%%%%%%%%%%%%%%%%%%%%%%%%%%%%%%%%%%%%%%%%%%%%%%%%%%%%%%%%%%%%%%%%%%%%%%%%%%%%%%%%%%%%
%%%%%%%%%%%%%%%%%%%%%%%%%%%%%%%%%%%%%%%%%%%%%%%%%%%%%%%%%%%%%%%%%%%%%%%%%%%%%%%%%%%%%%
%%%%%%%%%%%%%%%%%%%%%%%%%%%%%%%%%%%%%%%%%%%%%%%%%%%%%%%%%%%%%%%%%%%%%%%%%%%%%%%%%%%%%%
%%%%%%%%%%%%%%%%%%%%%%%%%%%%%%%%%%%%%%%%%%%%%%%%%%%%%%%%%%%%%%%%%%%%%%%%%%%%%%%%%%%%%%
\section{Faktorisierungsproblem}
\label{sec:Faktorisierungsproblem}
Für die Schlüsselgenerierung in RSA müssen zwei Primzahlen gefunden werden. Mit Hilfe eines Primzahltests (z.B. Miller-Rabin-Test) kann man prüfen, ob es sich bei den zwei Zahlen um eine Primzahl handelt.\\
Das Multiplizieren von $p$ und $q$ (mit $p\perp q$) um an $p\cdot q=m$ ranzukommen ist leicht. Aber um aus $m$ an $p$ und $q$ ranzukommen, muss eine Faktorisierung in Form einer Primfaktorzerlegung an $m$ durchgeführt werden. Da $m$ ein Produkt zweier Primzahlen ist, wird die Primfaktorzerlegung folgendes Ergebnis liefern:
\begin{equation*}
\label{formel:m_aus_Primzahlen}
m=p\cdot q
\end{equation*}
Das Problem bei einer Primfaktorzerlegung (bzw. Faktorisierung) ist, dass bisher kein effizientes Verfahren bekannt ist, um die Primfaktorzerlegung einer beliebigen Zahl zu erhalten.\\
Im Folgenden werden zwei bekanntes Verfahren vorgestellt, um eine Primfaktorzerlegung durchzuführen:
\\
\\
\underline{Primfaktorzerlegung durch Probedivision}\\
Bei der Probedvision handelt es sich um das einfachste Verfahren zur Zerlegung einer Zahl $m$ in seine Primfaktoren. Bei diesem Verfahren wird wiederholt \glqq probiert\grqq, ob $m$ durch Primzahlen teilbar ist.\\
Folgendes Beispiel mit $x=126$ zeigt das Vorgehen bei der Probedivision:
\\
\\
% \textbf{Schritt 1:} \\
% Die nächst größere Quadratzahl $q$ zu $m=126$ finden:
% \begin{align*}
% q&=\Big\lceil \sqrt{m}\Big\rceil^{2}=a^{2}\\
% \Rightarrow q&=\Big\lceil \sqrt{126}\Big\rceil^{2}=12^{2}=144
% \end{align*} 
% \\
% Die Zahl $a=12$ gibt an, welche Primzahlen ausprobiert werden müssen, um $m$ in ihre Primfaktoren zu zerlegen. Es sind alle Primzahlen $P \in \mathbb{P}$, für die $2\leqq p \leqq a$ gilt:\\
% \begin{equation*}
% P=\{2,3,5,7,11\}
% \end{equation*}
\textbf{Schritt 1:} \\
% Zuerst wird $m$ so oft durch die \glqq erste\grqq Primzahl 2 geteilt, bis dies nicht mehr möglich ist. Dann wird $m$ so oft durch die \glqq nächst größere\grqq Primzahl 3 geteilt, bis dies nicht mehr möglich ist. Darauf folgen die 5, 7, usw., bis die zu teilende Zahl eine Primzahl ist. \\
% \\
Zuerst wird $x$ überprüft, ob sie eine Primzahl ist (z.B. mit dem Miller-Rabin Test):
\begin{center}
$x=126$ ist eine gerade Zahl. Also kann sie keine Primzahl sein.
\end{center}
Da $x$ keine Primzahl ist, kann sie faktorisiert werden. Zuerst wird sie so oft mit der \glqq ersten\grqq~Primzahl 2 geteilt, bis es nicht mehr geht:
\begin{equation*}
126/2=63\\
\end{equation*}
Die 63 ist nicht mehr durch 2 teilbar. Also wird versucht, die Zahl durch die \glqq nächst größere\grqq~Primzahl 3 zu teilen und so lange zu wiederholen, bis es nicht mehr geht:
\begin{align*}
63/3&=21\\
21/3&=7
\end{align*}
Da die 7 eine Primzahl ist\footnote{Bei großen Zahlen kann am Ende auch eine sehr große Zahl stehen. Bei Unsicherheit sollte diese Zahl z.B. mit dem Miller-Rabin Test geprüft werden, ob es sich wirklich um eine Primzahl handelt.}, ist sie der letzte Primfaktor von $x=126$. Insgesamt wurde $m$ einmal durch 2, zweimal durch 3 und einmal durch 7 geteilt. Daraus folgt:\\
\begin{equation*}
126=2\cdot3\cdot3\cdot7=2\cdot3^{2}\cdot7\\
\end{equation*}
Damit ist die Primfaktorzerlegung durch Probedivision abgeschlossen\footnote{vgl. Wikipedia (a) (2015)}.\\
Die Probedivision stößt allerdings schnell an ihre Grenzen, wenn $x$ ein Produkt aus zwei großen Primzahlen $p$ und $q$ ist, da jede Primzahl von 2 bis $q$ (im Fall dass $p>q)$ durchprobiert werden muss. Eine etwas effektivere Methode für diesen Fall bietet die Faktorisierungsmethode von Fermat.\\ 
\\
\underline{Primfaktorzerlegung mit der Faktorisierungsmethode von Fermat}\\
Die Faktorisierungsmethode von Fermat beruht darauf, dass jede ungeradene Zahl $m$ als Differenz zweier Quadratzahlen dargestellt werden kann.

\begin{Faktorisierung}[Jede ungeradene ganze Zahl ist eine Differenz zweier Quadratzahlen]
Jede ungerade ganze Zahl $m$ lässt sich mit einer geeigneten geraden Zahl $k$ als\\
folgendes schreiben:
\begin{equation*}
m=2\cdot k+1
\end{equation*}
\textbf{\underline{Behauptung:}}\\
Die Zahl $m$ lässt sich auch als Differenz zweier aufeinanderfolgenden Quadratzahlen\\ schreiben:
\begin{equation*}
m=(k+1)^{2}-k^{2}
\end{equation*}
\textbf{\underline{Beweis der Behauptung:}}
\begin{align*}
m&=(k+1)^{2}-k^{2} = k^{2}+2\cdot k+1-k^{2}\\
&=2\cdot k+1
\end{align*}
Hiermit ist bewiesen, dass jede ungeradene ganze Zahl $m$ als Differenz zweier Quadratzahlen geschrieben werden kann:
\begin{align*}
m=2\cdot k+1=(k+1)^{2}-k^{2} \qquad q.e.d
\end{align*}
\end{Faktorisierung}
Ist $m$ ein Produkt zweier Primzahlen lässt sie sich in zwei Primzahlen faktorisieren:
\begin{equation*}
m=p\cdot q
\end{equation*}
Dank der 3. Binomischen Formel lässt sich folgendes definieren:
\begin{align*}
\label{eq:3_bin_formel_Definition}
p&:=(a+b)\\
q&:=(a-b)\\
m=p\cdot q&=(a+b)\cdot(a-b)\\
\Rightarrow m&=a^{2}-b^{2}
\end{align*}
Die Zahl $m$ ist nun die Differenz zweier Quadratzahlen. Es müssen nun $a$ und $b$ bestimmt werden. Sind diese beiden Zahlen bestimmt, lässt sich daraus $p$ und $q$ ableiten und man hätte $m$ faktorisiert.\\
Die folgende Beispielrechnung erläutert nun die Faktorisierungsmethode von Fermat:\\
\\
\underline{Beispiel:}\\
Es wird angenommen, dass $m$ ein Produkt der Primzahlen $p=727$ und $q=619$ ist.\\
Daraus folgt:
\begin{equation*}
m=450013
\end{equation*}
Dank der 3. Binomischen Formel (siehe vorheriger Abschnitt) gilt folgendes:
\begin{align*}
&m=a^{2}-b^{2}\\
\Leftrightarrow~&a^{2}-m=b^{2} 
\end{align*}
Nun wird die Faktorisierungsmethode von Fermat ausgeführt.\\
\\
\textbf{Schritt 1:} Bestimmen der nächst größeren Quadratzahl $Q$ von $m$:
\begin{align*}
Q&=\Big\lceil \sqrt{m}\Big\rceil^{2}=a^{2}\\
\Rightarrow Q&=\Big\lceil \sqrt{450013}\Big\rceil^{2}=671^{2}=450241
\end{align*}
\textbf{Schritt 2:} Differenz prüfen:\\
Wenn die Differenz $d=a^{2}-m$ eine Quadratzahl $b^{2}$ ist, lassen sich daraus die Faktoren $p$ und $q$ von $m$ bestimmen. Ist $d$ keine Quadratzahl, wird der Schritt mit der nächst höheren Quadratzahl durchgeführt:
\begin{align*}
% (a &=a)   &:\qquad &d=a^{2} &-n      &=b^{2}?\quad &|~&Keine~Quadratzahl\\
% (a &=671) &:\qquad &d=450241&-450013 &=228\quad &|~&Keine~Quadratzahl\\
% (a &=672) &:\qquad &d=451584&-450013 &=1571\quad &|~&Keine~Quadratzahl\\
% (a &=673) &:\qquad &d=452929&-450013 &=2916\quad &|~&Quadratzahl
(a &=a):   &  d &= a^{2}  - m       & &=b^{2}~? & &|~Quadratzahl?\\
\midrule
(a &=671): &  d &= 450241 - 450013  & &=228     & &|~Keine~Quadratzahl\\
(a &=672): &  d &= 451584 - 450013  & &=1571    & &|~Keine~Quadratzahl\\
(a &=673): &  d &= 452929 - 450013  & &=2916    & &|~Quadratzahl
\end{align*}
Nun sind $a^{2}$ und $b^{2}$ bestimmt worden:
\begin{align*}
a^{2} &= 452929 =673^{2}\\
b^{2} &= 2916 =54^{2}
\end{align*}
\textbf{Schritt 3:} Die (Prim-)Faktoren $p$ und $q$ bestimmen:
\begin{align*}
m &= a^{2}-b^{2} = (a+b) \cdot (a-b)\\
  &= 673^{2}-54^{2} = (673+54) \cdot (673-54)\\
  &= 727 \cdot 619\\
\Rightarrow m &=p \cdot q = 727 \cdot 619
\end{align*}
Damit ist die Primfaktorzerlegung durch die Faktorisierungsmethode von Fermat abgeschlossen\footnote{vgl. Wikipedia (b) (2015)}.\\
Seine Stärke kann diese Methode ausspielen, wenn $p$ und $q$ sich nicht stark voneinander unterscheiden (nur eine kleine Differenz haben) und somit nahe $\sqrt{m}$ liegen.\\
\\
\underline{Zur Verdeutlichung:}
\begin{itemize}
\item Je kleiner die Differenz von $p$ und $q$ ist, desto kleiner ist $b^{2}$ \quad \textit{(siehe \underline{Schritt 3})}
\item Je kleiner $b^{2}$ ist, desto kleiner ist die Differenz $d=a^{2}-m$ \quad \textit{(siehe \underline{Schritt 2})}
\item Je kleiner die Differenz $d$ ist, desto weniger Quadratzahlen $q$ bzw. $a^{2}$ müssen \glqq durchprobiert\grqq~werden
\item Je weniger Quadratzahlen ausprobiert werden müssen, desto schneller kommt diese Faktorisierungsmethode zum Ziel
\end{itemize}  
\textbf{Zusammenfassend: Erkenntnisse aus den beiden Verfahren zur Primfaktorzerlegung}\\
Anhand der Primfaktorzerlegung durch Probedivision lässt sich erkennen, dass es für die Auswahl der zwei Primzahlen $p$ und $q$ zur Schlüsselgenerierung in RSA sehr wichtig ist, hohe Zahlen\footnote{Die Literatur empfiehlt Primzahlen in der Größenordnung von ca. $10^{200}$} zu nehmen. Ansonsten lässt sich das Produkt der zwei Primzahlen sehr schnell durch Probedivision faktorisieren, da nicht so viele Primzahlen \glqq durchprobiert\grqq~werden müssen.\\
\\
Die Faktorisierungsmethode von Fermat liefert ebenfalls eine wichtige Erkenntnis für die Auswahl der Primzahlen zur Schlüsselgenerierung:\\
Die beiden Primzahlen müssen sich stark voneinander unterscheiden (eine große Differenz haben), um nicht leicht faktorisiert werden zu können, da sie ansonsten zu nahe an $\sqrt{m}$ liegen.
Liegen beide Primzahlen zu nahe an $\sqrt{m}$, kann $m$ sehr schnell durch die Faktorisierungsmethode von Fermat faktorisiert werden.
%%%%%%%%%%%%%%%%%%%%%%%%%%%%%%%%%%%%%%%%%%%%%%%%%%%%%%%%%%%%%%%%%%%%%%%%%%%%%%%%%%%%%%
%%%%%%%%%%%%%%%%%%%%%%%%%%%%%%%%%%%%%%%%%%%%%%%%%%%%%%%%%%%%%%%%%%%%%%%%%%%%%%%%%%%%%%
%%%%%%%%%%%%%%%%%%%%%%%%%%%%%%%%%%%%%%%%%%%%%%%%%%%%%%%%%%%%%%%%%%%%%%%%%%%%%%%%%%%%%%
%%%%%%%%%%%%%%%%%%%%%%%%%%%%%%%%%%%%%%%%%%%%%%%%%%%%%%%%%%%%%%%%%%%%%%%%%%%%%%%%%%%%%%
%%%%%%%%%%%%%%%%%%%%%%%%%%%%%%%%%%%%%%%%%%%%%%%%%%%%%%%%%%%%%%%%%%%%%%%%%%%%%%%%%%%%%%
\section{Eulersche Phi-Funktion}
\label{sec:Euler_Phi_Funktion}
\begin{Euler}[Eulersche $\varphi$-Funktion]
\label{def_Eulerfunktion}
Die Eulersche Phi-Funktion (auch Eulersche $\varphi$-Funktion oder Eulersche Funktion) gibt die Anzahl aller Zahlen $a \in \mathbb{N}$ an, die zu $n \in \mathbb{N}$ teilerfremd und nicht größer als $n$ sind.
\begin{equation*}
\varphi(n):=\Big| \{a \in \mathbb{N}\} \big|~ggT(a,n)=1 \land 1 \le a \le n \} \Big|
\end{equation*}
\end{Euler}
Zwei natürliche Zahlen $a$ und $n$ sind teilerfremd, wenn der größte gemeinsame Teiler der beiden Zahlen die 1 ist ($ggT(a,n)=1$) oder wenn beide Zahlen keinen gemeinsamen Primfaktor haben.\\
\textbf{Als Beispiel}\\
Die Zahl $n=6$ hat zwei Zahlen, die die in der Definition genannten Bedingungen erfüllen. Nämlich $a_1=1$ und $a_2=5$:
\begin{itemize}
\item \textbf{Für} $\mathbf{a_1=1}$\textbf{:} Die Bedingungen $ggT(1,6)=1$ und $1\le1\le6$ treffen zu
\item \textbf{Für} $\mathbf{a_2=5}$\textbf{:} Die Bedingungen $ggT(5,6)=1$ und $1\le5\le6$ treffen zu
\end{itemize}
Demnach sind $a_1$ und $a_2$ zu 6 teilerfremd und auch nicht größer als 6. Daraus folgt:
\begin{equation*}
\varphi(6)=2
\end{equation*}
\textbf{Berechnung}\\
Bei asymmetrischen Kryptosystemen wird die $\varphi$-Funktion nur auf Primzahlen angewendet. Da eine Primzahl $p$ nur durch sich selbst und die 1 teilbar ist, sind alle Zahlen von 1 bis $(p-1)$ teilerfremd zu $p$. Daher lässt sich die $\varphi$-Funktion einer Primzahl wie folgt berechnen:
\begin{equation*}
\varphi(p)=p-1
\end{equation*}
Da die $\varphi$-Funktion eine multiplikative\footnote{Eine zahlentheoretische Funktion heißt \textit{multiplikativ}, wenn für teilerfremde Zahlen $a$ und $b$ stets\\$f(ab)=f(a)\cdot f(b)$ gilt.} Funktion ist, gilt für zwei Primzahlen $p$ und $q$ folgendes:
\begin{equation*}
\varphi(p\cdot q)=\varphi(p)\cdot \varphi(q)=(p-1)\cdot(q-1)
\end{equation*}
\underline{Beispiel für $\varphi(15)$:}\\
$\varphi(15)=\varphi(5\cdot3)=\varphi(5)\cdot\varphi(3)=(5-1)\cdot(3-1)=8$
\\
\\
Eine wichtige Anwendung findet die $\varphi$-Funktion in RSA bei der Generierung des Public- und Private Keys\footnote{vgl. Steinfield (2015)}.
%%%%%%%%%%%%%%%%%%%%%%%%%%%%%%%%%%%%%%%%%%%%%%%%%%%%%%%%%%%%%%%%%%%%%%%%%%%%%%%%%%%%%%
%%%%%%%%%%%%%%%%%%%%%%%%%%%%%%%%%%%%%%%%%%%%%%%%%%%%%%%%%%%%%%%%%%%%%%%%%%%%%%%%%%%%%%
%%%%%%%%%%%%%%%%%%%%%%%%%%%%%%%%%%%%%%%%%%%%%%%%%%%%%%%%%%%%%%%%%%%%%%%%%%%%%%%%%%%%%%
%%%%%%%%%%%%%%%%%%%%%%%%%%%%%%%%%%%%%%%%%%%%%%%%%%%%%%%%%%%%%%%%%%%%%%%%%%%%%%%%%%%%%%
%%%%%%%%%%%%%%%%%%%%%%%%%%%%%%%%%%%%%%%%%%%%%%%%%%%%%%%%%%%%%%%%%%%%%%%%%%%%%%%%%%%%%%
% \section{Satz von Euler-Fermat}
% \label{sec:Fermat_Euler}
% Der Satz von Euler-Fermat sagt folgendes aus:
% \begin{equation*}
% a^{\varphi(n)} \equiv 1~(mod~n)
% \end{equation*}
% Dieser Satz wird benötigt, um zu beweisen, dass RSA funktioniert.
%%%%%%%%%%%%%%%%%%%%%%%%%%%%%%%%%%%%%%%%%%%%%%%%%%%%%%%%%%%%%%%%%%%%%%%%%%%%%%%%%%%%%%
%%%%%%%%%%%%%%%%%%%%%%%%%%%%%%%%%%%%%%%%%%%%%%%%%%%%%%%%%%%%%%%%%%%%%%%%%%%%%%%%%%%%%%
%%%%%%%%%%%%%%%%%%%%%%%%%%%%%%%%%%%%%%%%%%%%%%%%%%%%%%%%%%%%%%%%%%%%%%%%%%%%%%%%%%%%%%
%%%%%%%%%%%%%%%%%%%%%%%%%%%%%%%%%%%%%%%%%%%%%%%%%%%%%%%%%%%%%%%%%%%%%%%%%%%%%%%%%%%%%%
%%%%%%%%%%%%%%%%%%%%%%%%%%%%%%%%%%%%%%%%%%%%%%%%%%%%%%%%%%%%%%%%%%%%%%%%%%%%%%%%%%%%%%
\section{Moderner Euklidischer Algorithmus}
\label{sec:Euklid_Algorithm}
Der moderne Euklidische Algorithmus (\textit{im Folgenden vereinfacht als Euklidischer Algorithmus bezeichnet}) spielt beim RSA-Kryptosystem eine wichtige Rolle, da mit ihm der $ggT$ zweier Zahlen bestimmt werden kann. Mit ihm lässt sich also auch prüfen, ob eine ausgesuchte Zahl zu einer anderen Zahl teilerfremd ist. Außerdem ist der Euklidischer Algorithmus die Grundlage des erweiterten Euklidischen Algorithmus, der zum Berechnen eines multiplikativen Inverse einer Zahl in einem Restklassenring verwendet wird.\\
\\
\textbf{Funktionsweise des Euklidischen Algorithmus}\\
Der euklidische Algorithmus basiert auf der Division mit Rest. In jedem Schritt des Algorithmus wird eine solche Division ausgeführt.
Soll also der $ggT$ von zwei Zahlen $a$ und $b$ bestimmt werden, wird wie folgt vorgegangen:\\
Der Algorithmus beginnt mit einer Division von  $a\div r_{0}$ mit $r_1$ als Rest und $b=r_{0}$:
\begin{equation*}
a=q_1\cdot r_0 + r_1
\end{equation*}
In jedem weiteren Schritt wird mit dem Divisor und dem Rest des vorherigen Schritts wieder eine Division mit Rest durchgeführt. Das wird so lange gemacht, bis der Rest Null ist, also die Division aufgegangen ist:
\begin{align*}
r_0 &= q_2\cdot r_1 + r_2\\
r_1 &= q_3\cdot r_2 + r_3\\
&\vdots\\
r_{n-1} &= q_{n+1}\cdot r_n + 0
\end{align*}
Das $r_n$ der letzten Division ist der größte gemeinsame Teiler\footnote{vgl. Serlo, o.J}:
\begin{equation*}
ggT(a,b)=r_n
\end{equation*}
\underline{Beispiel:}\\
Ein Beispiel für den Euklidischen Algorithmus wird im Kapitel (\textit{\nameref{sec:Erweitert_Euklid}} auf Seite \pageref{euklid_beispiel}) vorgestellt.
%%%%%%%%%%%%%%%%%%%%%%%%%%%%%%%%%%%%%%%%%%%%%%%%%%%%%%%%%%%%%%%%%%%%%%%%%%%%%%%%%%%%%%
%%%%%%%%%%%%%%%%%%%%%%%%%%%%%%%%%%%%%%%%%%%%%%%%%%%%%%%%%%%%%%%%%%%%%%%%%%%%%%%%%%%%%%
%%%%%%%%%%%%%%%%%%%%%%%%%%%%%%%%%%%%%%%%%%%%%%%%%%%%%%%%%%%%%%%%%%%%%%%%%%%%%%%%%%%%%%
%%%%%%%%%%%%%%%%%%%%%%%%%%%%%%%%%%%%%%%%%%%%%%%%%%%%%%%%%%%%%%%%%%%%%%%%%%%%%%%%%%%%%%
%%%%%%%%%%%%%%%%%%%%%%%%%%%%%%%%%%%%%%%%%%%%%%%%%%%%%%%%%%%%%%%%%%%%%%%%%%%%%%%%%%%%%%
\section{Multiplikatives Inverse im Restklassenring}
\label{sec:Inverse_Restklassenring}
Die multiplikativ inversen Elemente im Restklassenring sind sozusagen der Schlüssel zum Erfolg des RSA-Verfahren (siehe Kapitel \textit{\nameref{sec:Beweis}} auf Seite \pageref{sec:Beweis}).\\
\\
\underline{Zum Grundverständnis:}\\
Ein Element $x^{-1}$ ist das multiplikativ inverse Element von $x$, wenn folgende Bedingung erfüllt ist:
\begin{equation*}
 x \cdot x^{-1}=1
\end{equation*}\\
\underline{In einem Restklassenring sieht das multiplikativ inverse Element wie folgt aus:}\\
Ein Element $a^{-1} \in \mathbb{Z}_n$ ist das multiplikativ inverse Element von $a \in \mathbb{Z}_n$, wenn folgende Bedingung erfüllt ist\footnote{vgl. Spiegelberg (2011)}:
\begin{equation*}
a \cdot a^{1} \equiv 1 \quad mod(n)
\end{equation*}
\underline{\textbf{Lemma von Bézout}}\\
Das Lemma besagt folgendes\footnote{Wikipedia (c) (2015)}:\\
Der $ggT(a,b)$ lässt sich als Linearkombination von $a$ und $b$ mit ganzzahligen Koeffizienten darstellen:
\begin{equation*}
\forall~a,b \in \mathbb{Z}~\exists~s,t \in \mathbb{Z}~|~ggT(a,b)=s\cdot a+t\cdot b
\end{equation*}
Beim RSA-Verfahren muss das multiplikativ inverse Element $d$ von $e~mod(\varphi(N))$ bestimmt werden. Da $e$ teilerfremd zu $\varphi(N)$ ist, ist deren $ggT(e,\varphi(N))=1$.\\
Nach dem Lemma von Bézout gilt also folgendes:
\begin{equation*}
1\equiv d \cdot e + k \cdot \varphi(N) \quad mod(\varphi(N))
\end{equation*}
Da $d$ das multiplikativ inverse Element von $e$ ist, gilt auch folgendes:
\begin{align*}
d\cdot e &\equiv 1 \quad mod(\varphi(N))\\
\Rightarrow k \cdot \varphi(N) &\equiv 0 \quad mod(\varphi(N))
\end{align*}
Das $k$ dient beim RSA Verfahren nur als \glqq Hilfsvariable\grqq und wird nicht weiter benötigt.\\
Weil das Lemma von Bézout gilt, kann mit dem erweiteren euklidischen Algorithmus eine Linearkombination aus $e$ und $\varphi(N)$ dargestellt werden, womit das inverse Element $d$ bestimmt werden kann.
%%%%%%%%%%%%%%%%%%%%%%%%%%%%%%%%%%%%%%%%%%%%%%%%%%%%%%%%%%%%%%%%%%%%%%%%%%%%%%%%%%%%%%
%%%%%%%%%%%%%%%%%%%%%%%%%%%%%%%%%%%%%%%%%%%%%%%%%%%%%%%%%%%%%%%%%%%%%%%%%%%%%%%%%%%%%%
%%%%%%%%%%%%%%%%%%%%%%%%%%%%%%%%%%%%%%%%%%%%%%%%%%%%%%%%%%%%%%%%%%%%%%%%%%%%%%%%%%%%%%
%%%%%%%%%%%%%%%%%%%%%%%%%%%%%%%%%%%%%%%%%%%%%%%%%%%%%%%%%%%%%%%%%%%%%%%%%%%%%%%%%%%%%%
%%%%%%%%%%%%%%%%%%%%%%%%%%%%%%%%%%%%%%%%%%%%%%%%%%%%%%%%%%%%%%%%%%%%%%%%%%%%%%%%%%%%%%
\section{Erweiterter Euklidischer Algorithmus}
\label{sec:Erweitert_Euklid}
Wie im vorherigen Kapitel gesagt, kann der erweiterter euklidischer Algorithmus angewendet werden, um die multiplikativ inversen Elemente in einem ganzzahligen Restklassenring zu berechnen. Dies wird bei der Bestimmung des privaten Schlüssels im RSA-Kryptosystems benötigt, da es unter anderem durch das multiplikative Inverse zu einem Element des öffentlichen Schlüssels bestimmt wird.\\
\\
Der erweiterter euklidischer Algorithmus wird wie folgt zur Bestimmung eines multiplikativen inversen Elements angewendet:\\
Wenn das multiplikativ inverse Element von $e~mod (\varphi(N))$ bestimmt werden soll, muss zuerst der euklidische Algorithmus an $e$ und $\varphi(N)$ angewendet werden. Dieser produziert eine Folge von Divisionen mit Rest. Anschließend kann man diese Divisionsgleichungen \glqq rückwärts lesen\grqq~und den Rest jeweils als Differenz der anderen Divisionsgleichungen darstellen. Dies wird rekursiv durchgeführt, bis man alle Divisionsgleichungen eingesetzt hat.\\
\\
% \underline{Beispiel zur Verdeutlichung:}\\
% \label{euklid_beispiel}
% Es soll das multiplikativ inverse Element $d$ von $e~mod(\varphi(N))$ mit $e=11$ und $\varphi(N)=24$ bestimmt werden. Es gilt also:
% \begin{align*}
% d \cdot e &\equiv 1 \quad mod(\varphi(N))\\
% \Rightarrow d \cdot 11 &\equiv 1 \quad mod(24)
% \end{align*}
% Nach dem Lemma von Bézout kann dies so dargestellt werden:
% \begin{equation*}
% 1 \equiv d \cdot 11 + k \cdot 24 \quad mod(24)
% \end{equation*}
% Nun wird der euklidische Algorithmus ausgeführt:
% \begin{align*}
% 24 &= 2 \cdot 11 + 2\\
% 11 &= 5 \cdot 2 + 1\\
% 2 &= 2 \cdot 1 + 0
% \end{align*}
% Jetzt kann der erweiterte euklidischer Algorithmus ausgeführt werden:\\
% Man betrachtet die vorletzte Zeile der Divisionsgleichungen. Da ist der Rest 1.\\
% Laut Lemma von Bézout soll der Rest 1 auf der einen Seite der Gleichung stehen. Deswegen wird diese Zeile nun umgeformt:
% \begin{align*}
% 11 &= 5 \cdot 2 + 1\\
% \Rightarrow 1 &= 11-5\cdot 2
% \end{align*}
% Nun wird \glqq rückwärts gelesen\grqq~und der Rest jeweils als Differenz der anderen Divisionsgleichungen dargestellt. Dies wird rekursiv durchgeführt, bis man alle Divisionsgleichungen eingesetzt hat:
% \begin{align*}
% 1 &= 11-5\cdot 2\\
%   &= 11-5\cdot (24-2\cdot 11)\\
%   &= 11-5\cdot 24+10\cdot 11\\
%   &= -5\cdot24+11\cdot 11
% \end{align*}
% Betrachtet man nun die letzte Gleichung im Restklassenring $\mathbb{Z}_{24}$ ergibt sich folgende Gleichung:
% \begin{equation*}
% 1 \equiv-5\cdot24+11\cdot 11 \quad mod(24)
% \end{equation*}
% \centering{mit}
% \begin{align*}
% 0 &\equiv -5\cdot 24 \quad mod(24)\\
% 1 &\equiv 11\cdot 11 \quad mod(24)\\
% \end{align*}
\underline{Beispiel zur Verdeutlichung:}\\
\label{euklid_beispiel}
Es soll das multiplikativ inverse Element $d$ von $e~mod(\varphi(N))$ mit $e=983$ und\\
$\varphi(N)=986040$ bestimmt werden. Es gilt also:
\begin{align*}
d \cdot e &\equiv 1 \quad mod(\varphi(N))\\
\Rightarrow d \cdot 983 &\equiv 1 \quad mod(986040)
\end{align*}
Nach dem Lemma von Bézout kann dies so dargestellt werden:
\begin{equation*}
1 \equiv d \cdot 983 + k \cdot 986040 \quad mod(24)
\end{equation*}
Nun wird der euklidische Algorithmus durchgeführt:
\begin{align*}
986040 &= 1003 \cdot 983 + 91\\
983 &= 10 \cdot 91 + 73\\
91 &= 1 \cdot 73 + 18\\
73 &= 4 \cdot 18 + 1\\
18 &= 18 \cdot 1 + 0
\end{align*}
Jetzt kann der erweiterte euklidischer Algorithmus ausgeführt werden:\\
Man betrachtet die vorletzte Zeile der Divisionsgleichungen. Da ist der Rest 1.\\
Laut Lemma von Bézout soll der Rest 1 auf der einen Seite der Gleichung stehen. Deswegen wird diese Zeile nun umgeformt:
\begin{align*}
73 &= 4 \cdot 18 + 1\\
\Rightarrow 1 &= 73-4\cdot 18
\end{align*}
Nun wird \glqq rückwärts gelesen\grqq~und der Rest jeweils als Differenz der anderen Divisionsgleichungen dargestellt. Dies wird rekursiv durchgeführt, bis man alle Divisionsgleichungen eingesetzt hat:
\begin{align*}
1 &= 73-4\cdot 18\\
  &= 73-4\cdot (91-1\cdot 73)\\
  &= 73-4\cdot91+4\cdot73\\
  &= -4\cdot91+5\cdot73\\
  &= -4\cdot91+5\cdot(983-10\cdot91)\\
  &= -4\cdot91+5\cdot983-50\cdot91\\
  &= 5\cdot983-54\cdot91\\
  &= 5\cdot983-54\cdot(986040-1003\cdot983)\\
  &= 5\cdot983-54\cdot986040+54162\cdot983\\
  &= 54167\cdot983-54\cdot986040
\end{align*}
Betrachtet man nun die letzte Gleichung im Restklassenring $\mathbb{Z}_{986040}$ ergibt sich folgende Gleichung:
\begin{equation*}
1 \equiv54167\cdot983-54\cdot 986040 \quad mod(986040)
\end{equation*}
mit den Produkten:
\begin{align*}
0 &\equiv -54\cdot 986040 \quad mod(986040)\\
1 &\equiv 54167\cdot 983 \quad mod(986040)\\
\end{align*}
Da $1 \equiv 54167\cdot 983~mod(986040)$ gilt, muss $54167$ das multiplikative inverse Element von $983~mod(986040)$ und somit $d$ sein\footnote{vgl. Spiegelberg (2011)}.
%%%%%%%%%%%%%%%%%%%%%%%%%%%%%%%%%%%%%%%%%%%%%%%%%%%%%%%%%%%%%%%%%%%%%%%%%%%%%%%%%%%%%%
%%%%%%%%%%%%%%%%%%%%%%%%%%%%%%%%%%%%%%%%%%%%%%%%%%%%%%%%%%%%%%%%%%%%%%%%%%%%%%%%%%%%%%
%%%%%%%%%%%%%%%%%%%%%%%%%%%%%%%%%%%%%%%%%%%%%%%%%%%%%%%%%%%%%%%%%%%%%%%%%%%%%%%%%%%%%%
%%%%%%%%%%%%%%%%%%%%%%%%%%%%%%%%%%%%%%%%%%%%%%%%%%%%%%%%%%%%%%%%%%%%%%%%%%%%%%%%%%%%%%
%%%%%%%%%%%%%%%%%%%%%%%%%%%%%%%%%%%%%%%%%%%%%%%%%%%%%%%%%%%%%%%%%%%%%%%%%%%%%%%%%%%%%%
\section{Kleiner Beweis des Konzepts hinter RSA}
\label{sec:Beweis}
Um beweisen zu können, dass das RSA-Verfahren funktioniert, muss bewiesen werden, dass eine Nachricht $t$ mit dem Public-Key verschlüsselt und mit dem Private-Key wieder entschlüsselt werden kann und die originale Nachricht $t$ erhalten bleibt.\\
Wird die Verschlüsselung als Funktion $V$ und die Entschlüsselung als Funktion $E$ definiert, muss demnach folgende Aussage bewiesen werden:\\
\begin{equation*}
E(V(t))=t
\end{equation*}
Beim RSA-Verfahren sehen die Funktion $V$ und $E$ so aus:
\begin{align*}
V(t)&=t^{e}~mod(N)=c\\
E(c)&=c^{d}~mod(N)=t
\end{align*}
Da allgemein $E(V(t))=t$ bewiesen werden muss, folgt daraus, dass hier folgendes bewiesen werden muss:
\begin{align*}
& E(V(t))=t\\
\Rightarrow & \Big[t^{e}~mod(N)\Big]^{d}~mod(N) \equiv t \quad mod(N)\\
\Rightarrow & (t^{e})^{d} \equiv t \quad mod(N)
\end{align*}
Als Bedingung ist in RSA folgende Aussage definiert:
\begin{align*}
& d~\text{ist ein multiplikativ inverses Element von}~e~mod(N)\\
\Rightarrow~ & d\cdot e \equiv 1 \quad mod(N)
\end{align*}
Daraus folgt:
\begin{equation*}
(t^{e})^{d} \equiv t^{e\cdot d} \equiv t^{1} \equiv t \quad mod(N)
\end{equation*}
Anhand dieses Beweises sieht man, dass die multiplikative Inversion der \glqq Schlüssel zum Erfolg\grqq~im RSA-Kryptosystem ist.
%%%%%%%%%%%%%%%%%%%%%%%%%%%%%%%%%%%%%%%%%%%%%%%%%%%%%%%%%%%%%%%%%%%%%%%%%%%%%%%%%%%%%%
%%%%%%%%%%%%%%%%%%%%%%%%%%%%%%%%%%%%%%%%%%%%%%%%%%%%%%%%%%%%%%%%%%%%%%%%%%%%%%%%%%%%%%
%%%%%%%%%%%%%%%%%%%%%%%%%%%%%%%%%%%%%%%%%%%%%%%%%%%%%%%%%%%%%%%%%%%%%%%%%%%%%%%%%%%%%%
%%%%%%%%%%%%%%%%%%%%%%%%%%%%%%%%%%%%%%%%%%%%%%%%%%%%%%%%%%%%%%%%%%%%%%%%%%%%%%%%%%%%%%
%%%%%%%%%%%%%%%%%%%%%%%%%%%%%%%%%%%%%%%%%%%%%%%%%%%%%%%%%%%%%%%%%%%%%%%%%%%%%%%%%%%%%%
\section{Nachtrag: Restklassenring als Voraussetzung der Einwegsfunktion in RSA}
\label{sec:Nachtrag}
Im RSA-Verfahren wird bei der Bestimmung des Private-Key im Restklassenring gerechnet. Dafür muss das inverse Element $d$ von $e$ im Restklassenring $\mathbb{Z}_{\varphi(N)}$ bestimmt werden.\\
\\
\underline{Die Frage ist nun:}\\
Warum muss das inverse Element im Restklassenring berechnet werden?
Laut des kleinen Beweises auf Seite \pageref{sec:Beweis} funktioniert das Verschlüsseln und Entschlüsseln in RSA aufgrund der Inversion der Elemente im Public- und Private-Key:
\begin{equation*}
(t^{e})^{d} \equiv t^{e\cdot d} \equiv t^{1} \equiv t \quad mod(N)
\end{equation*}
Demnach gilt auch Folgendes (mit der Bedingung $x\cdot y=1$):
\begin{equation*}
(t^{x})^{y} = t^{x\cdot y} = t^{1} = t
\end{equation*}
Damit wäre der Public-Key=($x,N$) und der Private-Key=($y,N$).\\
Da $y$ das inverse Element von $x$ ist, gilt:
\begin{equation*}
y=\frac{1}{x}
\end{equation*}
Daraus folgt:
\begin{align*}
\text{Public-Key} &=(x,N)\\
\text{Private-Key} &=(\frac{1}{x},N)
\end{align*}
Wie zu sehen, ist es kein Problem, aus dem Public-Key den Private-Key zu bilden und stellt somit keine Einwegsfunktion mit Falltür dar, da der Private-Key ohne Weiteres aus dem Public-Key gebildet werden kann.\\
\\
Wird ein inverses Element hingegen im Restklassenring $\mathbb{Z}_{\varphi(N)}$ bestimmt\footnote{mit dem erweitertem Euklidischem Algorithmus}, wird $\varphi{(N)}$ benötigt.
$\varphi(N)$ lässt sich allerdings nur effizient bestimmen, wenn die Primfaktoren von $N$ bekannt sind. Wie im Kapitel \nameref{sec:Faktorisierungsproblem} auf Seite \pageref{sec:Faktorisierungsproblem} erläutert, lässt sich $N$ derzeit noch nicht in vertretbarer Zeit in seine Primfaktoren zerlegen.\\
Somit ist es für das RSA-Verfahren unabdingbar, dass multiplikativ inverse Elemente in Restklassen berechnet werden. Nur so bleibt das Verschlüsselungsverfahren eine Einwegsfunktion.
\author{Autor: Patrick Künzl}
\chapter{\ac{RSA}}
\section{Motivation}
Online-Zahlungen sind aus unserem Alltag nichtmehr wegzudenken. Aus Paypals
letztem Quartalsbericht für das Jahr 2015 konnte man entnehmen, dass allein am
"`Cyber Monday"' 450 Zahlungen pro Sekunde verarbeitet wurden. Das sind circa
1,6 Millionen Geldbewegungen in einer Stunde.\footnote{Vgl. \cite{MotleyFool}}
Für Hacker demnach ein Paradies, um Nutzerdaten oder Geldflüsse zu beeinflussen.
Um dieses Szenario zu vermeiden, werden z.B. Online-Transaktionen durch
verschiedene Verschlüsselungsverfahren (auch Kryptosysteme genannt) gesichert.
Eines dieser Kryptosysteme ist das "`RSA-Verfahren"'. Es ist eines der
ältesten Verfahren und in der Kategorie der \emph{asymmetrischen
Verschlüsselungsverfahren} einzuordnen. RSA kommt z.B.
bei E-Mail-Verschlüsselungen (OpenPGP, S/MIME) oder der Telefonie zum Einsatz.
Dieses Kapitel beschäftigt sich mit dem Konzept/ der Funktionsweise von RSA,
seiner mathematischen Durchführung/Besonderheiten (anhand eines Beispiels
dargestellt) und der Sicherheit.
\section{Einleitung}
Das RSA-Verfahren wurde von Ron \textbf{R}ivest, Adi \textbf{S}hamir und Leonard \textbf{A}dlemann
entwickelt. 1977 entstand die erste öffentliche Version des
Verschlüsselungsverfahrens.\footnote{Vgl.
\cite{RSAWiki}} Es ist verwandt mit dem "`Rabin-Kryptosystem"'. RSA basiert auf
2 wichtigen Grundideen:
\begin{enumerate}
  \item Public-Key Verschlüsselung: \newline
  Der Kerngedanke besteht, anders als bei den symmetrischen
  Verschlüsselungsverfahren, daraus, 2 verschiedene Schlüssel für eine
  Verschlüsselung zu verwenden. Die Schlüssel zum
  Verschlüssen sind öffentlich, während der
  Schlüssel zum Entschlüsseln privat ist und nicht weitergegeben werden darf.
  Das Public-Key-Verfahren besteht formal aus drei verschiedenen Algorithmen:
  \begin{itemize}
    \item Dem Schlüsselerzeugungsalgorithmus, welcher die Grundlage für die
    Verschlüsselung schafft und den öffentlichen und privaten Schlüssel
    generiert.
    \item Dem Verschlüsselungsalgorithmus, der mithilfe des öffentlichen
    Schlüssels den Klartext in einen Geheimtext umwandelt. Ein Klartext kann
    hierbei in mehrere Geheimtexte resultieren, je nachdem welchen öffentlichen
    Schlüssel man benutzt. $1:n$
    \item Dem Entschlüsselungsverfahren, welcher aus dem Geheimtext mittels
    des privaten Schlüssels den Klartext berechnet. Hier ist nur $1:1$ Zuordnung
    möglich, das bedeutet, dass ein Geheimtext nur mit einem bestimmten privaten
    Schlüssel entschlüsselt werden kann.
  \end{itemize}
  \item Digitale Signatur: \newline
	Das RSA-Verfahren kann ebenso zur Signierung von Nachrichten benutzt werden.
	Hierbei wird die Klartext-Nachricht des Senders mit dem Private-Key
	verschlüsselt und mit dem öffentlichen Schlüssel des Empfängers entschlüsselt.
	Mit dem privaten Schlüssel eine Nachricht zu "`verschlüsseln"' ist keinerlei
	ein Widerspruch unter der Prämisse, dass $T = V$ (Klartext gleich Geheimtext)
	ist.
	\begin{displaymath}
	E(T) = C \ \ \ \ \ \footnote{Vgl. \cite{RSAAlgorithm} - S.
  1}
	\end{displaymath}
	\begin{center}
	\emph{E = Verschlüsseln (Encrypt); T = Text; C = Ciphertext (Geheimtext) }
	\end{center}
	Die Zahlen, welche die Nachricht repräsentieren unterscheiden keinen Klartext
	von einem Geheimtext. Die Interpretation des Lesers wiederum entscheidet, ob
	der Text einen logischen Zusammenhang ergibt (dann ist es der sogenannte
	"`Klartext"') oder nicht (dann ist es der Geheimtext). Durch die
	Verschl+sselung mit dem privaten Schlüssel wird nun erreicht, dass wenn der
	Empfänger empfangenen Text entschlüsselt, er eine lesbare Nachricht zu Augen
	bekommt.
	\begin{displaymath}
	D(C) = D(E(T)) = T \ \ \ \ \ \footnote{Vgl. \cite{RSAAlgorithm} - S.
  2}
	\end{displaymath}
	\begin{center}
	\emph{E = Verschlüsseln (Encrypt); D = Entschlüsseln (Decrypt); T = Text; C =
	Ciphertext (Geheimtext) }
	\end{center}
	Schickt der Sender gleichzeitig die selbe Nachricht unverschlüsselt zum
	Empfänger, hat der Empfänger die Möglichkeit beide Nachrichten miteinander zu
	vergleichen.
	Sollten diese übereinstimmen, kann der Empfänger sicher sein, dass diese
	Nachricht von dem Inhaber des privaten Schlüssels gesendet worden ist. Die
	Integrität und Authentizität wird somit garantiert.
\end{enumerate}

Das Geheimrezept für eine starke Verschlüsselung bildet ein schwieriges
mathematisches Problem. Das bedeutet, dass man den Hinweg (Zahlen werden zu
einem Ergebnis verarbeitet) einer Berechnung einfach durchführen kann, aber der
Rückweg (vom Ergebnis zu den verwendeten Zahlen) sich als komplexes Problem
darstellt (Stichwort: \emph{Einwegfunktionen}). Ein
solches komplexes Problem wäre die Primfaktorzerlegung, auf dessen das
RSA-Verfahren beruht. Bei der Primfaktorzerlegung werden 2 verschiedene
Primzahlen miteinander multipliziert, dies ist der einfache Teil. Aus dem
Produkt beide Primzahlen zu ermitteln, gestaltet sich jedoch als
weitaus schwieriger. Es gilt: "`Je größer der Multiplikator und der
Multiplikand, desto schwieriger die Ermittlung der einzelnen Bestandteile aus
dem Produkt"'. In der Praxis werden Primzahlen mit mehreren 100 Dezimalstellen
verwendet.
\section{Funktionsweise}
Folgende Grafik veranschaulicht die Funktionsweise/den Ablauf des
RSA-Verfahrens. \newline
\begin{figure}[H]
\includegraphics[width=1\textwidth]{publickey.png}
\caption[Ablauf des RSA-Verfahren]{grafische Darstellung des RSA-Verfahren -
Quelle: http://pajhome.org.uk/crypt/rsa/intro.html, Stand 18.02.2016}
\end{figure}

Im Folgenden werden die einzelnen Schritt ausführlich erläutert.
  \subsection{Die Schlüsselgenerierung}
  Bevor Dateien ver- und wieder entschlüsselt werden können, wird zunächst der
  öffentliche und private Schlüssel kreiert. Dazu wählt man zwei
  unterschiedlich große Primzahlen \emph{p} und \emph{q} und bildet das Produkt
  \emph{n}, auch bekannt als \emph{RSA-Modul}.
  \begin{displaymath}
  N = p * q
  \end{displaymath}
  Danach wird $\phi(N)$ berechnet.
  \begin{displaymath}
  \phi(N) = \phi(p) * \phi(q) = (p-1) * (q-1)
  \end{displaymath}
  Als nächsten Schritt wählt man eine Zahl $e$ mit $1<e<\phi(N)$, wobei der
  größte gemeinsame Teiler von $e$ und $\phi(N)$ 1 sein muss. ($ggT(e,
  \phi(N)))=1$)
  Der öffentliche Schlüssel setzt sich nun aus den Werten $e$ und $N$ zusammen.
  \begin{displaymath}
  KeyOeffentlich = (e,N)
  \end{displaymath}
  $p$, $q$ und $\phi(N)$
  dürfen nicht rausgegeben werden. \newline\newline
  Für den privaten Schlüssel bestimmt man $d$ mit
  \begin{displaymath}
  e*d \equiv 1 \ mod(\phi(N))
  \end{displaymath}
  und löst die Gleichung $e*d+k*\phi(N)\equiv 1 \ mod(\phi(N))$ mit dem
  erweiterten euklidischen Algorithmus. Das x, welches man aus der Tabelle
  errechnet (für eine anschauliche Darstellung siehe nächsten Abschnitt) ist nun
  das Inverse-Element von $e$. Demnach haben wir nun alle benötigten Elemente
  für unseren privaten Schlüssel zusammen.
  \begin{displaymath}
  KeyPrivat = (d, N)
  \end{displaymath}
  \subsection{Verschlüsselung}
  Möchte man nun einen Klartext mit dem öffentlichen Schlüssel verschlüsseln, so
  gilt die Formel:
  \begin{displaymath}
  V = T^e \ mod(N)
  \end{displaymath}
  \subsection{Entschlüsselung}
  Um die Umkehrfunktion der Verschlüsselung zu bilden, ersetzen wir $e$ mit $d$
  aus dem privaten Schlüssel. Daraus ergibt sich:
  \begin{displaymath}
  T = V^d \ mod(N)
  \end{displaymath}
\section{Praktisches Beispiel}
\emph{Damit man das folgende Beispiel besser nachvollziehen kann, sei kurz
gesagt, dass Daten nicht nur durch Bitfolgen angesehen werden können, sondern
auch durch Zahlen. Für den Computer macht es keinen Unterschied, ob er das Zeichen "`C"'
als Bitfolge abspeichert oder den ASCII-Code von C - "`67"'. Für die
Verschlüsselung macht es jedoch einen großen Unterschied, da man mit Zahlen
mathematisch agieren kann, im Gegensatz zu Buchstaben. Nachteil dieser
Herangehensweise ist die Tatsache, dass sowohl Sender als auch Empfänger
das gleiche "`Dictionary"' benötigen, um aus der Zahlenbasis wieder einen reinen
Text zu gestalten. (zum Beispiel die
ASCII-Tabelle)}\newline\newline Nehmen wir an Alice möchte Bob eine
verschlüsselte Nachricht zukommen lassen.
Ihr Klartext $T$ enthält den Buchstaben "`C"'. C ist der dritte Buschstabe im
Alphabet, also wählen wir die 3. Somit ergibt sich $T = 3$.
Als nächstes werden die Schlüssel generiert. Hierzu denken wir uns zunächst 2
verschiedene Primzahlen aus und multiplizieren sie miteinander um das
Produkt $N$ herauszubekommen.
\begin{displaymath}
p = 5, q = 7 \rightarrow N = p * q = 35
\end{displaymath}
Danach bestimmen wir $\phi(N)$. Das besondere hierbei ist, dass $N$ aus
2 Primzahlen besteht. Für Primzahlen gilt: $ \phi(p) = p-1$. Somit gilt für
unser Beispiel:
\begin{displaymath}
\phi(N) = \phi(p) * \phi(q) = (p-1)*(q-1) = 4*6 = 24
\end{displaymath}
Jetzt brauchen wir unser $e$. $e$ muss teilerfremd zu $\phi(N)$ sein. Wie im
Grundlagenteil erläutert, gilt für $e$ folgende Regel:
\begin{displaymath}
e \in \mathbb{N} \ | \ ggT(e,N) = 1 \ \ \gets \  \ 1<e<\phi(N)
\end{displaymath}
Das $e$ stellt immer eine Primzahl dar, wobei nicht jede Primzahl teilerfremd
zu $e$ sein muss. In diesem Beispiel wäre 3 für $e$ keine gültige Primzahl da
$24/3 = 8$. In diesem Fall wird $e$ auf den Wert $e = 11$ gesetzt. Das
$e$ sollte zudem nicht allzu klein gewählt sein. Trotz einer erheblichen
Einbußung der Performance des Algorithmuses, ist es sehr unsicher, Zahlen wie 3
oder 17, zu verwenden. Falls ein Angreifer versuchen sollte sich das $\phi(N)$ durch
zurückrechnen herzuleiten, so wird es für ihn nicht leichter. Ebenso sollte es
vermieden werden $p$ oder $q$ als $e$ zu benutzen.
\newline\newline
Der öffentliche Schlüssel wäre somit $KeyOeffentlich = (11, 35)$.
\newline\newline
Im nächsten Schritt wird nun $d$ bestimmt. $d$ muss das Inverse von $e \
mod(\phi(N))$ darstellen, damit die allgemeine Formel $e*d \equiv 1 \
mod(\phi(N))$ gilt.
$e$ wurde so ausgesucht, dass es genau ein Inverses gibt. Um das Element zu berechnen
benötigen wir hier den erweiterten euklidischen Algorithmus. Dieser wurde im
Einleitungsteil umfassend erklärt. Als diophantische Gleichung ergibt sich:
\begin{displaymath}
11*x + 24 * y = 1
\end{displaymath}
Den Rechenweg der Gleichung erkennen Sie in Abbildung 2.\newline
\begin{figure}[H]
\includegraphics[width=1\textwidth]{eEA.png}
\caption[erweiterter euklidischer Algorithmus (Beispiel)]{eigene Darstellung
des erweiterten euklidischen Algorithmus zur diophantischen Gleichung $11*x +
24*y = 1$}
\end{figure}
Das x aus der Tabelle bildet das Inverse-Element, welches wir benötigen. An
dieser Stelle können 2 Besonderheiten auftreten.
\begin{enumerate}
  \item Wie in unserem Fall spiegelt unser $d$ auch unser $e$ wieder. Dies ist
  nicht einem Rechenfehler geschuldet, sondern der Tatsache der kleinen
  Schlüssellänge unseres Beispiels (6 Bit). Ab einer größeren Schlüssellänge
  unterscheiden sich $d$ und $e$ mit wachsender Wahrscheinlichkeit. Das es sich
  um keinen Rechenfehler handelt erkennt man später daran, dass die
  Zwischenergebnisse vor dem Modulo-Rechnen 2 unterschiedliche Zahlen sind.
  \item Es könnte ebenso der Fall auftreten, dass das x nach dem
  erweitereten euklidischen Algorithmus negativ ist. In diesem Fall darf man
  den ermittelten Wert nocheinmal mod(N) rechnen, z.B. $-9 = 31 \ mod
  (40)$.\footnote{Vgl.
  http://www.onlinemathe.de/forum/RSA-Was-wenn-inverser-Schluessel-negativ-ist}
\end{enumerate}
Um zu überprüfen, ob man den erweiterten euklidischen Algorithmus richtig
verwendet hat, werden die Zahlen in die obige Gleichung eingesetzt.
\begin{displaymath}
11*11 + 24 * -5 = 121 - 120 = 1
\end{displaymath}
Unser privater Schlüssel laute demnach $KeyPrivat = (11,35)$.
\newline\newline
Nachdem wir nun unsere Schlüssel generiert haben, kann mit der Verschlüsselung
angefangen werden. Zur Verschlüsselung gilt die Formel $V = T^e \ mod(N)$, mit
der Bedingung das $T$ nicht größer als $N$ ist, sonst wäre das Modulo-Rechnen
beim Entschlüsseln nicht erfolgreich.
In diesem Beispiel laute die Verschlüsselungsformel wie folgt:
\begin{displaymath}
V = 3^{11} \ mod(35)
\end{displaymath}

Rechenweg:
\begin{displaymath}
3^{11} \ mod(35)
\equiv 3^3 *(3^2)^4 \ mod(35)
\equiv 27 * (9)^4 \ mod(35)
\equiv 27 * (11)^2 \ mod(35)
\equiv 27 * 121 \ mod(35)
\end{displaymath}
\begin{displaymath}
\equiv 27 * 16 \ mod(35)
\equiv 12
\end{displaymath}
\newline
Verschlüsselt ergibt der Klartext $T$ nun den Wert $V = 12$. Unser Buchstabe
"`C"' symbolisiert für einen Außenstehenden theoretisch nun den
Buchstaben "`L"'.
\newline\newline
Um zu zeigen, dass der private Schlüssel trotz gleichem $d$ und $e$ stimmt,
können wir den Geheimtext wieder entschlüsseln.
Die Formel dazu lautet $T = V^d \ mod(N)$.

\begin{displaymath}
T = 12^{11} \ mod(35)
\end{displaymath}

Rechenweg:
\begin{displaymath}
12^{11} \ mod(35)
\equiv 12^1 * 12^2 * (12^2)^4 \ mod(35)
\equiv 12*144*144^4 \ mod(35)
\equiv 12*4*4^4 \ mod(35)
\end{displaymath}
\begin{displaymath}
\equiv 48*(4^2)^2 \ mod(35)
\equiv 13*(16)^2 \ mod(35)
\equiv 13*256 \ mod(35)
\equiv 13*11 \ mod(35)
\end{displaymath}
\begin{displaymath}
\equiv 143 \ mod(35)
\equiv 3
\end{displaymath}

Man erkennt, dass unser Klartext $T = 3$ lautet und somit gleich unserer
Ausgangszahl ist. Hiermit wurde bewiesen, dass die Schlüssel korrekt sind und
das Ver-/Entschlüsseln erfolgreich war.

\section{Sicherheit des RSA-Verfahrens}
Nachdem die mathematischen Aspekte des RSA-Verfahrens aufgezeigt worden sind,
stellt sich natürlich die Frage, wie standhaft denn das Konzept gegenüber
Angriffen ist. Grundsätzlich gilt natürlich: "`Je länger die Schlüssellänge,
desto sicherer das gesamte Verfahren"`. Gängige Werte für das RSA-Verfahren sind
1024 oder 2048 Bit.\footnote{Vgl. \cite{KryptoIX} - S. 176} Generell kann man
sagen, dass das RSA Verfahren bei gewissenhafter und richtiger Implementierung
ein sehr sicheres Verfahren darstellt und bestens für Verschlüsselungen geeignet
ist. Die Kehrseite der Medallie ist die Tatsache, dass die besten Verfahren auch
die besten Hacker anlocken um sie zu knacken. An dieser Stelle seien 3
verschiedene Herangehensweisen dargestellt.
\subsection{Vollständige Schlüsselsuche}
Die vollständige Schlüsselsuche fällt unter die Kategorie der
"`Key-Only-Attacks"' (auf Deutsch: Key-Substitution-Angriffe). Dabei "`versucht
ein Angreifer, sich einen Signaturprüfschlüssel zu generieren, so dass ein
gegebenes Nachricht/Signatur-Paar eines anderen Benutzers auch bei der
Verifikation mit seinem Signaturprüfschlüssel \textit{gültig}
ausgibt."'\footnote{siehe \cite{SpringerKey} - S. 389} Bezogen auf RSA hieße
das, dass der Angreifer so lange einen Schlüssel erzeugen würde, bis aus einem
Geheimtext ein (logischer) Klartext werden würde. Betrachten wir den Aufwand
$O(n/2)$ dieser Suche bei einer Schlüssellänge von 256-Bit, müsste der Angreifer
2$^{255}$ Schlüssel im Durschschnitt (Average Case) generieren. Das
enstpricht ungefähr 10$^{76}$ verschiedenen Schlüsseln - zum Vergleich das
Universum ist 10$^{18}$ Sekunden alt. Für einen "`überschaubaren"' Zeitraum zur
Findung der Lösung, müsste man ungefähr 10$^{58}$ verschiedene Schlüssel
ausprobieren.\footnote{Vgl. \cite{KryptoIX} - S. 177}
\subsection{Faktorisierungsproblem}
Um den Aufwand der vollständigen Schlüsselsuche zu reduzieren, wäre es
sinnvoller die verwendeten Primzahlen $p$ und $q$ herauszufiltern. Aus Ihnen
wäre es möglich, ohne großen Aufwand, den privaten Schlüssel zu generieren. Eine
Methode dazu wäre das $N$ des öffentlichen Schlüssels wieder zu zerlegen. Denn
$N$ ist schließlich das Produkt beider Primzahlen.
Die sogenannte \emph{Primfaktorzerlegung oder der Faktorisierungsangriff}
besitzt jedoch einen Nachteil. Bei sehr großen Produkten von 2 verschiedenen
Primzahlen, ist der Algorithmus nicht effizient genug, das Problem in
annehmbarer Zeit zu lösen.
Der Weltrekord einer faktorisierten RSA-Primzahl liegt derzeit bei einem 768-Bit
Schlüssel, dass sind ungefähr 232 Dezimalstellen - die Zeitdauer dafür
entsprach knapp 2,5 Jahre.
(Zum Vergleich, ein Faktor des RSA-768 lautet: \newline
$347807169895689878604416984821269081770479498371376856891
2431388982883793878$\newline$002287614711652531743087737814467999489$)\footnote{siehe
\cite{768RSA} - S. 13}
Im Umkehrschluss bedeutet das, dass ein 1024-Bit Schlüssel die minimale
Schlüssellänge heutzutage sein sollte und man damit aber noch ausreichend
geschützt ist.
\subsection{TWINKLE und TWIRL}
Kurioser Weise schlug der RSA-Miterfinder Adi Shamir 1999 auf einer
Eurocrypt-Konferenz in Prag ein Konzept für eine Maschine dar, welche speziell
auf das Problem der Faktorisierung ausgerichtet sein sollte -
TWINKLE (The Weizmann Insitute Key Locating Engine).\footnote{Vgl.
\cite{TWINKLE}} Das besondere an dieser Maschine ist ihre Bauweise, die
keinesfalls einem Computer sondern einen elektronisch, optischen Apparat
ähnelte.
2003 stellte Shamir erneut eine solche Maschine mit einem Kollegen vor - den
TWIRL (The Weizmann Insitute relation Locator). Doch auch TWIRL scheint ebenso
wie TWINKLE bisher nur als Konzept zu existieren und noch nie gebaut worden zu
sein. Shamir empfiehlt trotzdem für sehr vertrauenswürdige Informationen nur
Schlüssellängen von mindestens 2048 Bit zu verwenden.
\subsection{Fazit}
Es gibt verschiedene Angriffspunkte die das RSA-Verfahren zur Verfügung stellt
und gewiss mehr als so manch anderes Verschlüsselungsverfahren (z.B.
Diffie-Hellmann). Dennoch ist die Verwendung von Einwegfunktionen bei einer hoch
gewählten Schlüssellänge so enorm aufwendig sie wieder rückgängig zu machen,
dass mit dem heutigen Stand der Technik es nicht in annehmbarer Zeit geknackt
werden kann. Ein ernstzunehmendes Problem wird es erst, sobald ein neues,
effizientes mathematisches Verfahren zur Faktorisierung bekannt wird. Dann muss
sich aber nicht nur das RSA um seine Sicherheit fürchten.
Als sichere Schlüssellänge gelten Werte ab 1024 Bit. Wer lieber zu viel als zu
wenig Sicherheit bei seiner Verschlüsselung anstreben möchte, der darf beruhigt
zur 4096 Bit-Variante greifen.
\section{Riemannsche Vermutung}
Ein großes Risiko für das RSA-Verfahren bietet der Beweis der \emph{Riemannschen
Hyptohese/Vermutung}.
Genau wie beim Faktorisieren, gestaltet sich die Suche nach neuen Primzahlen ebenso als ein
schwieriges Unterfangen, welches enorme Rechenkapazitäten in Anspruch nimmt.
Durch die Unregelmäßigkeit der Primzahlen, muss jede einzelne Zahl genauestens betrachtet
und geprüft werden, ob sie eine Primzahl darstellt. Ein System hinter der
Verteilung von Primzahlen scheint es nicht zu geben. Bislang gibt es nur eine Vermutung/Hypothese, welche von Bernhard Riemann niedergeschrieben
worden ist. Er beschäftigte sich ausführlich mit der Verteilung von Primzahlen
und stieß auf eine interessante Entdeckung.
Er knüpfte bei seinen Forschungen an die Entdeckungen von Euler 100 Jahre zuvor
an, der mit der Formel
\begin{displaymath}
\frac{2^x}{2^x-1}*\frac{3^x}{3^x-1}*\frac{5^x}{5^x-1}*\frac{7^x}{7^x-1}*\frac{11^x}{11^x-1}*\ldots
=\frac{\pi^2}{6}
\end{displaymath}
zum ersten Mal bewies, dass Primzahlen etwas mit den Strukturen von
naturwissenschaftlichen Objekten zu tun hat. Riemann schreib diese Funktion ein
bisschen um und bildete daraus die Zeta-Funktion.
\begin{displaymath}
\zeta(2)=\sum_{n=1}^{\infty} \frac{1}{n^2}
\end{displaymath}
Das erstaunliche an dieser Funktion ist nun, dass die Nullstellen der
Zeta-Funktion im dreidimensionalen Raum überall dort liegen, wo eine Primzahl
als Rechenzahl einbezogen worden ist. Noch erstaunlicher ist es, dass diese Nullstellen alle auf einer
einzigen Geraden liegen. Die Riemannsche Vermutung legt als die These fest, dass
alle existierenden Primzahlen auf dieser Geraden liegen müssten.\footnote{Vgl.
  https://www.youtube.com/watch?v=CaFoSTxkIvY - ab 12:15}
\newline
Die Riemannsche Vermutung wurde auf die Liste der Millenium-Probleme gesetzt und zahlreiche
namenhaften Wissenschaftler haben schon versucht diese These zu beweisen. Bis jetzt ist es jedoch
noch keinem Gelungen. Sollte die Hypothese jemals beweisen werden, werden
Primzahlen als Verschlüsselungsmethode nie wieder Verwendung finden dürfen. Die
Kryptographie müsste dann einen Rundumschlag erleben.
\section{Schlusswort/Zusammenfassung}
RSA ist ein sehr starker Verschlüsselungsalgorithmus, welcher vielen Tests
bisher standgehalten hat. Es vereinbart viele praktische Anwendungsmöglichkeiten
und arbeitet trotz seiner komplexen Vorgehensweise relativ perfomant und
zuverlässig. Das enthaltene mathematische Problem der Faktorisierung bietet durch die langen Lösungszeiten einen guten
Puffer, um den Algorithmus noch weitere Jahre stabil zu halten. Sollte früher
als erwartet oder erhofft eine Lösung des Problems auftauchen, könnte dies eine
große Gefährdung des gesamten digitalen Zahlungsverkehrs und der Geheimhaltung
von vertraulichen Informationen bedeuten.
\newpage

% -*- LaTeX-command: pdflatex; TeX-master: "main.tex"; -*-
\author{Autor: Vasilij Schneidermann}
\chapter{Hashing}

Die Abbildung beliebiger Eingabedaten auf Ausgabedaten fester Größe
wird allgemein als \emph{Hashing} bezeichnet.  Das konkrete Verfahren
ist eine \emph{Hashfunktion}, das Ergebnis des Verfahrens ist der
\emph{Hashwert}\footnote{S.~207, Kryptographie}.

Typische Anwendungsfälle für Hashing sind \emph{Hash Tables} und
\emph{Sets} für die schnelle Suche eines Wertes in einem Wörterbuch,
bzw.~einer Menge, aber auch Authentifizierung und die Sicherstellung
von Integrität eines Dokuments in der Kryptographie.  Für letztere
sind \emph{kryptographische Hashfunktionen} notwendig.

\section{Hashfunktionen}

Die in der Einleitung verwendete Beschreibung kann unter der Annahme,
dass $H$ die Hashfunktion, $h$ der Hashwert und $m$ die zu hashende
Nachricht ist folgendermaßen ausgedrückt werden:

$$h = H(m)$$

Anhand der Anforderung, dass der Wertebereich für $m$ unendlich groß
ist, aber der Wertebereich für $h$ eine endliche Größe hat, kann
geschlussfolgert werden, dass $H$ \emph{nicht} injektiv sein kann.
Eine andere fundamentale Feststellung ist, dass durch diese Abbildung
nicht jede Nachricht einen einzigartigen Hashwert haben kann.
Vorkommnisse verschiedener Nachrichten mit dem gleichen Hashwert
werden als \emph{Kollisionen} bezeichnet\footnote{S.~382,
  IT-Sicherheit}.

Kollisionen sind als solches bei nicht-kryptographischen
Anwendungsfällen eher ein Ärgernis welches z.B.~bei einer Hash Table
gesondert behandelt wird und zu leicht verminderter Performance führt,
bei kryptographischen Anwendungsfällen hingegen werden Verfahren
bevorzugt welche das Risiko von auftretenden Kollisionen
minimieren\footnote{S.~382, IT-Sicherheit}.

Zur Vermeidung von Kollisionen werden üblicherweise zwei weitere
Anforderungen an die Hashfunktion hinzugezogen.  Einerseits sollen
möglichst alle möglichen Hashwerte erreichbar sein oder in anderen
Worten, die errechneten Hashwerte gleichverteilt sein, andererseits
ist es wünschenswert dass selbst minimale Änderungen an der Nachricht
zu vollkommen verschiedenen Hashwerten führen\footnote{S.~208,
  Kryptographie}.

\section{Beispiele nicht-kryptographischer Hashfunktionen}

Falls die Hashfunktion auf einen relativ kleinen Bereich von
Nachrichten angewandt wird, ist es möglich den Wert der Nachricht
selbst als Hashwert zu nutzen.  Dies wird zum Beispiel für das Hashing
vom Datentyp \emph{Integer} im OpenJDK
genutzt\footnote{\url{http://grepcode.com/file/repository.grepcode.com/java/root/jdk/openjdk/6-b14/java/lang/Integer.java\#Integer.hashCode\%28\%29}}.

Für das Hashing von Strings verwendet man etwas komplexere
Algorithmen.  Ein solches Beispiel wurde von Daniel J.~Bernstein
veröffentlicht\footnote{\url{https://groups.google.com/forum/\#!msg/comp.lang.c/VByoIO8GySs/2XN9iGTpgmsJ}},
dieses kann mit wenig Aufwand in C implementiert werden\footnote{\url{http://www.cse.yorku.ca/~oz/hash.html}}:

\begin{lstlisting}[language=C,basicstyle=\ttfamily\footnotesize,keywordstyle=\bfseries\color{black},captionpos=b,caption={djb2}]
unsigned long
hash(unsigned char *str)
{
    unsigned long hash = 5381;
    int c;

    while (c = *str++)
        hash = ((hash << 5) + hash) + c; /* hash * 33 + c */

    return hash;
}
\end{lstlisting}

\section{Kryptographische Hashfunktionen}

Für die Nutzung von Hashfunktionen in der Kryptographie kommen noch
weitere Anforderungen hinzu.  Da Kollisionen bekanntermaßen
existieren, sollte es für einen Angreifer nicht praktikabel sein eine
Kollision zu finden\footnote{S.~209, Kryptographie}.  Dabei wird
zwischen \emph{schwacher Kollisionsresistenz} und \emph{starker
  Kollisionsresistenz} unterschieden.

\subsection{Schwach kollisionsresistente Hashfunktionen}

Gegeben sei die bisherige Definition einer Hashfunktion.  Es müssen
zusätzlich folgende Kriterien erfüllt werden\footnote{S.~382"=383,
  IT-Sicherheit}:

\begin{enumerate}
\item $H$ ist eine Einwegfunktion, d.h.~es ist leicht $h = H(m)$ zu
  berechnen, aber schwer $m = H^{-1}(h)$ zu berechnen
\item Ist $h = H(m)$ gegeben, ist es schwierig eine andere Nachricht
  $m'$ zu finden für welche $h = H(m')$ gilt.
\end{enumerate}

Ist diese Eigenschaft nicht gegeben, ist es für einen Angreifer
möglich eine Nachricht zu seinem Gunsten zu fälschen welche den
gleichen Hashwert aufweist.

\subsection{Stark kollisionsresistente Hashfunktionen}

Folgende Kriterien müssen erfüllt werden\footnote{S.~385}:

\begin{enumerate}
\item $H$ muss eine schwach kollisionsresistente Hashfunktion sein
\item Es ist schwierig $m$ und $m'$ zu finden für welche $H(m) =
  H(m')$ gilt.
\end{enumerate}

\section{Beispiele kryptographischer Hashfunktionen}

\subsection{\ac{MD5}}

MD5 ist aus historischen Gründen eine noch weitverbreitete
kryptographische Hashfunktion, sollte aber aufgrund von
Sicherheitsbedenken nicht mehr eingesetzt werden, da es möglich ist in
relativ kurzer Zeit Kollisionen auf handelsüblicher Hardware zu
finden\footnote{S.~224, Kryptographie}.

\subsection{\ac{SHA1}}

SHA1 ist eine sehr populäre kryptographische Hashfunktion welche unter
anderem in SSL, IPsec und PGP verwendet wird\footnote{S.~221,
  Kryptographie}.  Jedoch wird an der Sicherheit gezweifelt, da es
mehrere Angriffe gegeben hat welche es ermöglichen könnten Kollisionen
herbeizuführen.

\subsection{\ac{SHA2}}

SHA2 ist eine Familie an SHA1-Varianten deren Sicherheit noch unklar
bleibt.  Aufgrunddessen werden sie in der Praxis selten
eingesetzt\footnote{S.~222-223, Kryptographie}.

\subsection{\ac{SHA3}}

SHA3 ist der neueste Standard der SHA-Familie welcher die Schwächen
der bisherigen Varianten vermeidet.  In einem internationalen
Wettbewerb wurde der Keccak-Algorithmus Mitte 2015 auserwählt und ist
damit die Empfehlung für sicheres
Hashing\footnote{\url{http://www.nist.gov/itl/csd/201508_sha3.cfm}}.

\section{Bekannte Attacken}

\subsection{Geburtstagsattacke}

Das Geburtstagsparadoxon beschreibt das Phänomen, dass es schon in
einer relativ kleinen Gruppe von Personen mit hoher Wahrscheinlichkeit
zwei mit dem gleichen Geburtstag geben wird.

Ist $k$ die Anzahl an möglichen Geburtstage, kann man die Anzahl $n$
an mindestens benötigten Personen für eine Wahrscheinlichkeit größer
$0.5$ durch $\sqrt{k}$ annähern\footnote{S.~213, Kryptographie}.

Die Formel kann auf Hashwerte ebenfalls angewendet werden.  Wenn
dieser eine feste Länge $x$ hat, beträgt $k$ dann $2^x$.  $n$ wäre in
diesem Fall also $2^{\frac{x}{2}}$, d.h.~durch das
Geburtstagsparadoxon wird die Schlüssellänge effektiv halbiert.  Ein
Angreifer müsste also in der Lage sein sowohl diese Anzahl an
Hashwerten errechnen sowie speichern zu können.  Deswegen verwendet
man in der Praxis mehr als die doppelte Schlüssellänge von dem was ein
Angreifer bewältigt bekommen würde, im Falle von SHA1 z.B.~160 Bits.

\subsection{Substitutionsattacke}

Kennt der Angreifer den Hashwert einer Nachricht die er zu seinem
Gunsten fälschen möchte, kann er eine andere Nachricht konstruieren,
Stellen finden welche er gefahrlos ersetzen kann ohne das Aussehen
oder die Bedeutung der Nachricht zu ändern und alle Kombinationen
möglicher Änderungen hashen.  Bei $n$ Änderungen wären dies $2^n$
denkbare Kombinationen.  Ist die Anzahl an Kombinationen hinreichend
hoch, ist es möglich eine gefälschte Nachricht mit dem gleichen
Hashwert wie die originale Nachricht zu finden.  Die
Wahrscheinlichkeit dafür ist besonders hoch wenn die Schlüssellänge
$x$ des Hashwertes kleiner $n$ ist, d.h.~es ist ratsam Hashwerte mit
hoher Schlüssellänge zu nutzen\footnote{S.~211"=212, Kryptographie}.

\subsection{Wörterbuchattacke}

Wird eine kurze Nachricht gehasht (wie z.B.~ein Passwort), ist eine
Wörterbuchattacke möglich.  Anstatt alle denkbaren Nachrichten
auszuprobieren, verwendet man ein Wörterbuch und probiert seine
Einträge durch.  Dieser Angriff ist erstaunlich effektiv in der
Praxis, da eine Menge Benutzer nicht auf sichere Passwörter setzen.

Eine mögliche Maßnahme zur Erhöhung des Angreiferaufwands ist das
\emph{Salt}, ein zusammen mit jedem einzelnen Hashwert gespeicherter
Wert welcher zusammen mit der Nachricht gehasht wurde.  Während diese
Vorgehensweise es für den Angreifer nicht schwieriger macht einen
einzelnen Hashwert zu knacken, führt sie dazu, dass er nicht mehrere
Hashwerte parallel knacken kann, da für jeden ein anderes Salt
verwendet wird\footnote{S.~214, Kryptographie}.

\emph{Key Stretching} ist eine andere solche Maßnahme.  Anstatt den
Hashwert an sich zu nehmen, wird er mit einer hinreichend sicheren
Anzahl an Vorgängen wiederholt bearbeitet.  Die einfachste Variante
davon ist das Resultat der Hashfunktion erneut als Eingabe zu nutzen.
Dadurch werden Angreifer gezwungen für jeden Versuch die gleiche
Anzahl an Hash-Iterationen zu verwenden\footnote{S.~215,
  Kryptographie}.

\subsection{Rainbow Tables}

Legt der Angreifer eine persistente Tabelle von sortierten Nachrichten
und ihren Hashwerten an, kann er diese verwenden um bekannte Hashwerte
schnell nachzuschlagen.  Auf diese Weise werden Bruteforce-Attacken
auf kurze Passwörter gangbar, eine Kombination mit Wörterbuchattacken
ist ebenfalls denkbar.  Salts und Key Stretching sind wirksame
Abhilfen, da man durch sie größere Rainbow Tables erfordert welche
womöglich nicht mehr praktikabel für den Angreifer
sind\footnote{S.~215"=217, Kryptographie}.

\author{Autor: Vasilij Schneidermann}
\chapter{Signaturen}

Involvieren schwarze Magie.

\section{\ac{MAC}s}

\section{Signaturen mit \ac{MAC}s}

\section{Alternative Signaturverfahren}

%!TEX root = main.tex
\author{Autor: Leonie Schiburr}
\chapter{Zertifikate und Public-Key-Infrastructure}
% Im folgenden Abschnitt soll die sichere Kommunikation mittels Zertifikaten und einer Public Key Infrastructure und die Gründe für die Nutzung
% dieser Kommunikationsmittel näher beleuchtet werden.
Im folgenden Abschnitt soll die sichere Kommunikation mittels Zertifikaten und einer Public-Key-Infrastructure (PKI) näher beleuchtet werden. Außerdem sollen Gründe für die Nutzung
dieser Kommunikationsmittel genannt werden.
%%%%%%%%%%%%%%%%%%%%%%%%%%%%%%%%%%%%%%%%%%%%%%%%%%%%%%%%%%%%%%%%%%%%%%%%%%%%%
%%%%%%%%%%%%%%%%%%%%%%%%%%%%%%%%%%%%%%%%%%%%%%%%%%%%%%%%%%%%%%%%%%%%%%%%%%%%%
%%%%%%%%%%%%%%%%%%%%%%%%%%%%%%%%%%%%%%%%%%%%%%%%%%%%%%%%%%%%%%%%%%%%%%%%%%%%%
%%%%%%%%%%%%%%%%%%%%%%%%%%%%%%%%%%%%%%%%%%%%%%%%%%%%%%%%%%%%%%%%%%%%%%%%%%%%%
%%%%%%%%%%%%%%%%%%%%%%%%%%%%%%%%%%%%%%%%%%%%%%%%%%%%%%%%%%%%%%%%%%%%%%%%%%%%%
\section{Motivation}
Das in den vorherigen Kapiteln beschriebene Public-Key-Verschüsselungsverfahren ist für die Kommunikation im Internet am weitesten verbreitet. Obwohl diese (zeitlich) aufwendiger ist als die Symmetrische Verschlüsselung, wird diese bevorzugt. Denn bei einer symmetrischen Verschlüsselung bedeutet ein öffentlich werden des gemeinsamen, geheimen Schlüssels gegenüber Dritten, dass die gesamte Kommunikation abgefangen, manipuliert und missbraucht werden kann. Mit der Verwendung eines öffentlichen und eines privaten Schlüssels wird dieses Risiko minimiert.\footnote{vgl. Hochschule Trier B (2009)}
\\
\\
Die beste Verschlüsselung nützt allerdings nichts, wenn die Nutzer der Methode nicht vertrauen. Woher weiß ein Anwender, dass der öffentliche Schlüssel,
den er zum Versenden einer chiffrierten Nachricht an den Empfänger verwenden möchte, auch wirklich dem Empfänger gehört. Die Gefahr besteht, dass
es sich um einen \glqq falschen\grqq~Schlüssel eines Betrügers handeln könnte, der über diesen an Daten gelangen will.
\\
\\
Eine Möglichkeit, die Authentizität eines Schlüssels und die Identität des Eigentümers zu verifizieren, bieten Zertifikate und die Public-Key-Infrastructure.
Die folgenden Kapiteln sollen diese Wege der sicheren Kommunikation näher beleuchten.
%%%%%%%%%%%%%%%%%%%%%%%%%%%%%%%%%%%%%%%%%%%%%%%%%%%%%%%%%%%%%%%%%%%%%%%%%%%%%
%%%%%%%%%%%%%%%%%%%%%%%%%%%%%%%%%%%%%%%%%%%%%%%%%%%%%%%%%%%%%%%%%%%%%%%%%%%%%
%%%%%%%%%%%%%%%%%%%%%%%%%%%%%%%%%%%%%%%%%%%%%%%%%%%%%%%%%%%%%%%%%%%%%%%%%%%%%
%%%%%%%%%%%%%%%%%%%%%%%%%%%%%%%%%%%%%%%%%%%%%%%%%%%%%%%%%%%%%%%%%%%%%%%%%%%%%
%%%%%%%%%%%%%%%%%%%%%%%%%%%%%%%%%%%%%%%%%%%%%%%%%%%%%%%%%%%%%%%%%%%%%%%%%%%%%
\section{Zertifikate}
Die sichere Kommunikation im Internet ist davon abhängig, dass sowohl die Authentizität, die Vertraulichkeit als auch die Integrität der Kommunikation untereinander gewahrt werden können.\\
Mit Hilfe asymmetrischer Verschlüsselung (z.B. durch RSA) und die Verwendung von Zertifikaten, soll dies gewährleistet werden:\\
\\
\underline{\textbf{Authentizität:}}\\
Absender A verschlüsselt eine Nachricht mit seinem privaten Schlüssel. Empfänger B kann die Nachricht mithilfe des öffentlichen Schlüssels von A entschlüsseln und weiß somit sicher, dass die Nachricht von Absender A verschickt wurde. Absender A hat seine Nachricht durch die Nutzung seines privaten Schlüssel digital signiert und somit seine Authentizität gesichert. Die Nachricht kann jedoch von jedem entschlüsselt werden, der Zugang zum öffentlichen Schlüssel von Absender A hat.\\
\\
\underline{\textbf{Vertraulichkeit:}}\\
Um eine vertrauliche Nachricht zu übermitteln muss Absender A die Nachricht für Empfänger B mit dessen öffentlichen Schlüssel verschlüsseln. Die Nachricht kann anschließend nur noch mit dem privaten Schlüssel von Empfänger B wieder entschlüsselt werden. Da dieser nur im Besitz von Empfänger B ist und er diesen geheim hält, kann die Nachricht nur von B gelesen werden. Dadurch wird die Vertraulichkeit der Nachricht gewährleistet. Empfänger B kann jedoch nicht mit Sicherheit wissen, dass die gesendete Nachricht von Absender A stammt.\\
\\
Durch die Anwendung von beiden Schlüsseln kann sowohl die Vertraulichkeit der Nachricht, als auch die Authentizität gewährleistet werden.\\
\\
\underline{\textbf{Integrität:}}\\
Der Teil der Nachricht, der geheim übermittelt werden soll, wird mit dem öffentlichen Schlüssel von Empfänger B verschlüsselt. Anschließend wird die gesamte Nachricht mit dem privaten Schlüssel von A verschlüsselt und so digital signiert.
So kann Empfänger B sicher sein, dass er die Nachricht von A erhält und der vertrauliche Teil der Nachricht nur durch ihn mithilfe seines privaten Schlüssels gelesen werden kann. Durch dieses Verfahren kann die Integrität der Kommunikation gesichert werden.\\
\\
Um diese Kommunikation erfolgreich durchführen zu können, muss sich Absender und Empfänger jedoch sicher sein, dass der jeweilige öffentliche Schlüssel auch tatsächlich dem Kommunikationspartner gehört und der Schlüssel auch für die angewendete Verschlüsselung genutzt werden darf.\\
\\
Zertifikate stellen neben dem öffentlichen Schlüssel selbst Informationen bereit, mit denen sich die Authentizität des Schlüssels prüfen lässt. \footnote{vgl. Hochschule Trier B (2009)}
%%%%%%%%%%%%%%%%%%%%%%%%%%%%%%%%%%%%%%%%%%%%%%%%%%%%%%%%%%%%%%%%%%%%%%%%%%%%%
%%%%%%%%%%%%%%%%%%%%%%%%%%%%%%%%%%%%%%%%%%%%%%%%%%%%%%%%%%%%%%%%%%%%%%%%%%%%%
%%%%%%%%%%%%%%%%%%%%%%%%%%%%%%%%%%%%%%%%%%%%%%%%%%%%%%%%%%%%%%%%%%%%%%%%%%%%%
%%%%%%%%%%%%%%%%%%%%%%%%%%%%%%%%%%%%%%%%%%%%%%%%%%%%%%%%%%%%%%%%%%%%%%%%%%%%%
%%%%%%%%%%%%%%%%%%%%%%%%%%%%%%%%%%%%%%%%%%%%%%%%%%%%%%%%%%%%%%%%%%%%%%%%%%%%%
\subsection{Begriffsklärung}
Zertifikate enthalten Informationen, welche die Identität des Eigentümers eines öffentlichen Schlüssels (und dem dazu gehörigen geheimen, privaten Schlüssel), den Aussteller des Zertifikats sowie weitere Eigenschaften des Schlüssels und Informationen über das Zertifikat selbst beschreiben. Insbesondere sind dies Informationen, um die Authentizität des Zertifikats verifizieren zu können.\footnote{vgl. Hochschule Trier B (2009)}\\
\\
Im Allgemeinen ist der Begriff des Zertifikats eher neutraler gehalten und muss nicht direkt auf einen kryptographischen Schlüssel bezogen sein. In der Praxis spricht man jedoch in den meisten Fällen von einem Public-Key-Zertifikat, das auf einen öffentlichen Schlüssel (Public-Key) verweist. \footnote{vgl. Wikipedia A (2014)} Diese Public-Key-Zertifikate basieren vornehmlich auf dem Format X.509, aktuell in der Version V3. \footnote{vgl. IT-Wissen (2016)}\\
\\
Zudem gibt es sogenannte Attributzertifikate. Diese enthalten keinen öffentlichen Schlüssel. Sie verweisen auf das Public-Key-Zertifikat und legen dessen Geltungs- und Anwendungsbereich genauer fest. Dies ermöglicht die Zuordnung von einer oder mehrerer Eigenschaften in unterschiedlichen Anwendungsfällen ohne das Originalzertifikat verändern zu müssen. Neben dem Verweis auf das entsprechende Zertifikat können Attribute beispielsweise Informationen über die Qualifikation einer Person oder Nutzungsbeschränkungen des Zertifikats dieser Person enthalten. Hier ein Beispiel: Herr B., Leitender Angestellter des Vertriebs SoSo GmbH, darf mit diesem Zertifikat Transaktionen bis maximal 2000\euro{} durchführen.\\ 
\\
Zertifikate können auf verschiedene Arten ausgegeben werden, aber sollten möglichst von einer bekannten Instanz ausgestellt werden, der die Nutzer der Zertifikate vertrauen.\footnote{vgl. Wikipedia A (2014), vgl. Rouse, Margaret A (2015)}\\
\\
Die Bereitstellung, die Aktualisierung und der Widerruf von Zertifikaten wird durch diese Instanz, auch als Zertifizierungsstelle (Certification Authority, CA) bezeichnet, geregelt. Diese ist Teil der Public-Key-Infrastructur, welche die sichere Kommunikation im Internet mittels Zertifikaten bereitstellen kann und im Kapitel \textit{\nameref{sec:PKI}} näher erläutert wird.\footnote{vgl. IT-Wissen (2016)}
%%%%%%%%%%%%%%%%%%%%%%%%%%%%%%%%%%%%%%%%%%%%%%%%%%%%%%%%%%%%%%%%%%%%%%%%%%%%%
%%%%%%%%%%%%%%%%%%%%%%%%%%%%%%%%%%%%%%%%%%%%%%%%%%%%%%%%%%%%%%%%%%%%%%%%%%%%%
%%%%%%%%%%%%%%%%%%%%%%%%%%%%%%%%%%%%%%%%%%%%%%%%%%%%%%%%%%%%%%%%%%%%%%%%%%%%%
%%%%%%%%%%%%%%%%%%%%%%%%%%%%%%%%%%%%%%%%%%%%%%%%%%%%%%%%%%%%%%%%%%%%%%%%%%%%%
%%%%%%%%%%%%%%%%%%%%%%%%%%%%%%%%%%%%%%%%%%%%%%%%%%%%%%%%%%%%%%%%%%%%%%%%%%%%%
\subsection{Aufbau}
Public-Key-Zertifikate bestehen in der Regel aus den folgenden Komponenten:\footnote{vgl. Rouse, Margaret A (2015), vgl. Hochschule Trier B (2009)} 
\begin{itemize}
\item Den Namen oder eine andere eindeutige Bezeichnung des Ausstellers des Zertifikates,
\item eine Seriennummer des Zertifikats,
\item Informationen zu Verfahren und Regeln, die bei der Ausgabe des Zertifikats verwendet wurden,
\item die Angabe eines Zeitraums, in dem das Zertifikat gültig ist,
\item der öffentlichen Schlüssel, auf den das Zertifikat sich bezieht,
\item der Name oder eine andere eindeutige Bezeichnung des Besitzers des öffentlichen Schlüssels und gegebenenfalls weitere Informationen zum Besitzer,
\item Angaben zu Anwendungs- und Geltungsbereich des öffentlichen Schlüssels,
\item sowie die digitale Signatur des Ausstellers oder der ausstellenden Institution. Damit kann der Empfänger prüfen, ob das Zertifikat echt ist.
\end{itemize}
Zusätzliche Eigenschaften, die bestimmte Zusammenhänge betreffen, können in den bereits oben beschriebenen Attributzertifikaten genauer festgehalten werden.
Die Vertrauenswürdigkeit eines Zertifikats hängt davon ab, ob und wie schnell es gesperrt werden kann und wie zuverlässig und zeitnah die Sperrung veröffentlicht wird, sollte die Gültigkeit des Zertifikats erlöschen. Dies kann neben dem normalen Ablaufen der Gültigkeitsdauer auch aufgrund von Kompromittierung des Schlüsselmaterials, Ungültigkeit der Zertifikatsdaten oder Verlassen der Organisation, die das Zertifikat bereitstellt (bspw. PKI), sein.\footnote{vgl. Wikipedia C (2015)}
%%%%%%%%%%%%%%%%%%%%%%%%%%%%%%%%%%%%%%%%%%%%%%%%%%%%%%%%%%%%%%%%%%%%%%%%%%%%%
%%%%%%%%%%%%%%%%%%%%%%%%%%%%%%%%%%%%%%%%%%%%%%%%%%%%%%%%%%%%%%%%%%%%%%%%%%%%%
%%%%%%%%%%%%%%%%%%%%%%%%%%%%%%%%%%%%%%%%%%%%%%%%%%%%%%%%%%%%%%%%%%%%%%%%%%%%%
%%%%%%%%%%%%%%%%%%%%%%%%%%%%%%%%%%%%%%%%%%%%%%%%%%%%%%%%%%%%%%%%%%%%%%%%%%%%%
%%%%%%%%%%%%%%%%%%%%%%%%%%%%%%%%%%%%%%%%%%%%%%%%%%%%%%%%%%%%%%%%%%%%%%%%%%%%%
\subsection{Funktionsweise}
Zur Sicherstellung der Authentizität eines öffentlichen Schlüssels muss das dazugehörige Public-Key-Zertifikat geprüft werden. Dieses Zertifikat ist mit der digitalen Signatur des Ausstellers ausgestattet. Die Echtheit dieser Signatur kann mittels des öffentlichen Schlüssels des Ausstellers überprüft werden. Doch zur Verifizierung der Authentizität des öffentlichen Schlüssels des Ausstellers wird wiederum ein Zertifikat benötigt, welches wiederum mit einer digitalen Signatur versehen ist.\\
\\
Die Problematik besteht darin, dass diese Kette von digitalen Zertifikaten, auch Zertifizierungs- oder Validierungspfad genannt, theoretisch kein Ende hat. In der Regel wird dies dadurch gelöst, dass das \glqq letzte\grqq~Zertifikat von einer Zertifizierungsstelle ausgegeben wird, der alle Kommunikationspartner ohne weitere Prüfung vertrauen.\footnote{vgl. Rouse, Margaret A (2015)}\\
Die Infrastruktur, in der diese Kommunikation mit Zertifikaten und Prüfung dieser Zertifikate stattfindet, wird als Public-Key-Infrastructure bezeichnet.
Auf die Implementierung einer solchen Zertifikathierarchie wird im Kapitel \textit{\nameref{sec:Vertrauensmodelle}} näher eingegangen.
%%%%%%%%%%%%%%%%%%%%%%%%%%%%%%%%%%%%%%%%%%%%%%%%%%%%%%%%%%%%%%%%%%%%%%%%%%%%%
%%%%%%%%%%%%%%%%%%%%%%%%%%%%%%%%%%%%%%%%%%%%%%%%%%%%%%%%%%%%%%%%%%%%%%%%%%%%%
%%%%%%%%%%%%%%%%%%%%%%%%%%%%%%%%%%%%%%%%%%%%%%%%%%%%%%%%%%%%%%%%%%%%%%%%%%%%%
%%%%%%%%%%%%%%%%%%%%%%%%%%%%%%%%%%%%%%%%%%%%%%%%%%%%%%%%%%%%%%%%%%%%%%%%%%%%%
%%%%%%%%%%%%%%%%%%%%%%%%%%%%%%%%%%%%%%%%%%%%%%%%%%%%%%%%%%%%%%%%%%%%%%%%%%%%%
\subsection{Anwendung}
Die Public-Key-Verschlüsselung mit Zertifikaten wird in vielen Bereichen, vor allem bei der Kommunikation im Internet angewendet:
\begin{itemize}
\item Sicherheit in Netzwerkprotokollen, wie SSL mit HTTPS, IPsec und SSH
\item Schutz von E-Mail-Kommunikation beispielsweise durch Pretty Good Privacy (PGP): Pretty Good Privacy ist eine Software welche häufig zur Ver- und Entschlüsselung von E-Mails verwendet wird. Das Programm kann Daten sowohl symmetrisch als auch asymmetrisch verschlüsseln, Schlüssel erzeugen und diese verwalten.\footnote{vgl. Freund, Rosa (2004)}
\item Authentisierung und Zugriffskontrolle bei Chipkarten, wie Unternehmensausweisen: Viele Unternehmen geben Unternehmensausweise, sogenannte Smart Cards, an ihre Mitarbeiter aus. Mit diesen Ausweisen mit Chip erhalten die Mitarbeiter in der Regel nicht nur Zutritt aufs Gelände. Mit Hilfe der auf dem Chip gespeicherten Zertifizierungen können die Mitarbeiter auf diverse Dienste mit eingeschränktem Zugang zugreifen, wie bestimmte Intranet-Anwendungen mit eingeschränktem Zugriff, oder Gebäude-Abschnitte, die nur befugte Personengruppen betreten dürfen, wie Serverräume.\footnote{Eigene Erfahrung mit Telekom Unternehmensausweis}
\end{itemize}
Auf die genauen Anwendungsbeispiele wird im Rahmen dieser schriftlichen Ausarbeitung des Vortrag nicht näher eingegangen, da dies den Umfang der Arbeit übersteigen würde.\\
\\
Zertifikate können von unterschiedlichen Zertifizierungsstellen in verschiedener Qualität ausgegeben werden. Die Qualität der Zertifikate hängt davon ab, wie gründlich die Angaben der Antragssteller geprüft werden. Diese Informationen werden in den Zertifizierungsrichtlinien (Certification Policies, CP) der Aussteller festgelegt. Zertifikate mit eingeschränkter Sicherheit sind häufig schon kostenlos zu erwerben. Hier kann jedoch das Risiko bestehen, dass die persönlichen Angeben der Antragssteller nicht ausreichend geprüft werden. Andere Zertifizierungsstellen bestehen auf der Vorlage von offiziellen Papieren, wie dem Personalausweis, bevor eine Zertifizierung ausgegeben wird, um ein höheres Sicherheitsniveau zu erreichen. Zertifikate mit höheren Sicherheitsstandards sind meist nur kostenpflichtig zu erhalten.\\
\\
Bekannte Anbieter für Webserver- und E-Mail-Zertifikate sind beispielsweise GlobalSign und TeleSec der Deutschen Telekom. Anbieter, die vertrauenswürdige Zertifikate im Einklang mit dem deutschen Gesetz anbieten, sind beispielsweise verschiedene Bundesnotarkammern, die T-Systems/Deutsche Telekom AG oder D-TRUST Gateway-Zertifikate der Bundesdruckerei-Gruppe.\footnote{vgl. Wikipedia C (2015)}
%%%%%%%%%%%%%%%%%%%%%%%%%%%%%%%%%%%%%%%%%%%%%%%%%%%%%%%%%%%%%%%%%%%%%%%%%%%%%
%%%%%%%%%%%%%%%%%%%%%%%%%%%%%%%%%%%%%%%%%%%%%%%%%%%%%%%%%%%%%%%%%%%%%%%%%%%%%
%%%%%%%%%%%%%%%%%%%%%%%%%%%%%%%%%%%%%%%%%%%%%%%%%%%%%%%%%%%%%%%%%%%%%%%%%%%%%
%%%%%%%%%%%%%%%%%%%%%%%%%%%%%%%%%%%%%%%%%%%%%%%%%%%%%%%%%%%%%%%%%%%%%%%%%%%%%
%%%%%%%%%%%%%%%%%%%%%%%%%%%%%%%%%%%%%%%%%%%%%%%%%%%%%%%%%%%%%%%%%%%%%%%%%%%%%
\section{Public Key Infrastructure}
\label{sec:PKI}
Wie oben beschrieben kann die Authentizität eines öffentlichen Schlüssels über Zertifikate verifiziert werden. Die Vergabe dieser Zertifikate sollte jedoch einheitlich für alle Kommunikationspartner sein und nachvollziehbar gemacht werden. Auch die Überprüfung und Sperrung der Schlüssel und Zertifikate selbst muss realisiert werden können. Die Realisierung einer solchen Infrastruktur wird auch Public-Key-Infrastructure (PKI) genannt.
%%%%%%%%%%%%%%%%%%%%%%%%%%%%%%%%%%%%%%%%%%%%%%%%%%%%%%%%%%%%%%%%%%%%%%%%%%%%%
%%%%%%%%%%%%%%%%%%%%%%%%%%%%%%%%%%%%%%%%%%%%%%%%%%%%%%%%%%%%%%%%%%%%%%%%%%%%%
%%%%%%%%%%%%%%%%%%%%%%%%%%%%%%%%%%%%%%%%%%%%%%%%%%%%%%%%%%%%%%%%%%%%%%%%%%%%%
%%%%%%%%%%%%%%%%%%%%%%%%%%%%%%%%%%%%%%%%%%%%%%%%%%%%%%%%%%%%%%%%%%%%%%%%%%%%%
%%%%%%%%%%%%%%%%%%%%%%%%%%%%%%%%%%%%%%%%%%%%%%%%%%%%%%%%%%%%%%%%%%%%%%%%%%%%%
\subsection{Begriffsklärung}
Eine Public-Key-Infrastructure ist eine Infrastruktur, die das Ausstellen, Verteilen, Überprüfung und Sperrung von Zertifikaten ermöglicht.
Dabei stellt eine Zertifizierungsstelle (Certification Authority, CA) auf Antrag digitale Zertifikate zur Verfügung.\footnote{vgl. Rouse, Margaret B (2006)}\\
Da Zertifikate als eine Art digitales Versprechen für Authentizität funktionieren, ist das Vertrauen zwischen Nutzer und Aussteller des Zertifikats und die Art und Weise, wie dieses Vertrauensverhältnis geschlossen wird, wesentlich für die Implementierung einer PKI.\\
\\
In der Literatur spricht man bei PKIs in der Regel von hierarchischen PKIs (Hierarchical Trust), die auf der Basis von verschiedenen Zertifizierungshierarchien arbeiten. Im Folgenden wird der Aufbau einer PKI auch anhand dieses Modells erklärt. Weitere Formen von Vertrauensmodellen werden im Abschnitt \textit{\nameref{sec:Vertrauensmodelle}} erläutert.\footnote{vgl. Schmeh, Klaus (2001), S.284 f.}
%%%%%%%%%%%%%%%%%%%%%%%%%%%%%%%%%%%%%%%%%%%%%%%%%%%%%%%%%%%%%%%%%%%%%%%%%%%%%
%%%%%%%%%%%%%%%%%%%%%%%%%%%%%%%%%%%%%%%%%%%%%%%%%%%%%%%%%%%%%%%%%%%%%%%%%%%%%
%%%%%%%%%%%%%%%%%%%%%%%%%%%%%%%%%%%%%%%%%%%%%%%%%%%%%%%%%%%%%%%%%%%%%%%%%%%%%
%%%%%%%%%%%%%%%%%%%%%%%%%%%%%%%%%%%%%%%%%%%%%%%%%%%%%%%%%%%%%%%%%%%%%%%%%%%%%
%%%%%%%%%%%%%%%%%%%%%%%%%%%%%%%%%%%%%%%%%%%%%%%%%%%%%%%%%%%%%%%%%%%%%%%%%%%%%
\subsection{Aufbau}
Eine (hierarchische) Public-Key-Infrastructure ist aus den folgenden Komponenten aufgebaut:
\begin{description}
\item[Zertifizierungstelle / Zertifizierungsinstanz (Certification Authority, CA):]\hfil\\
Die Zertifizierungsstelle ist für die Ausgabe und Überprüfung der Public-Key-Zertifikate zuständig. Sie stellt die Zertifikate bereit und signiert die Zertifikate digital, um die Authentizität zu bestätigen.
\item[Registrierungsinstanz (Registration Authority, RA):]\hfill \\
Die Registrierungsinstanz bietet Personen, Unternehmen, untergeordnete Zertifizierungsstellen oder Ähnliches die Möglichkeit, Zertifikate zu beantragen. Die Registrierungsstelle verifiziert Zertifizierungsanträge, bevor diese von einer Zertifizierungsstelle ausgegeben werden. Die Registrierungsinstanz überprüft die Richtigkeit aller für ein Zertifikat angegebenen Daten und genehmigt den Zertifizierungsantrag. Die Zertifizierungsstelle signiert diesen im Anschluss und gibt das Zertifikat an den Antragssteller aus.
\item[Zertifikatssperrliste (Certificate Revocation List, CRL):]\hfill \\
Die Zertifikatssperrliste ist eine Liste mit Zertifikaten, die vor Ablauf der Gültigkeitdauer gesperrt beziehungsweise zurückgezogen wurden. Wenn Beispielsweise ein
privater Schlüssel entwendet wurde, wird das Schlüsselpaar gesperrt und in der Sperrliste vermerkt. So wird verhindert, dass der öffentliche Schlüssel weiterhin genutzt wird.
Eine Zertifikatssperrliste hat eine genau definierte Laufzeit. Nach Ablauf dieser Laufzeit muss die Sperrliste neu mit den aktuellen Daten generiert werden. Alternativ zur Zertifikatssperrliste kann auch eine Positivliste (White-List) angelegt werden, in der alle zum aktuellen Zeitpunkt gültigen Zertifikate aufgelistet sind.\\
\\
Eine Public Key Infrastructure muss für ihre Nutzer eine Statusprüfung für Zertifikate bieten. Neben der Zertifikatssperrliste und der Positiv-Liste kann auch eine Online-Statusprüfung durchgeführt werden. Eine Online-Prüfung ist in der Regel dann notwendig, wenn eine zeitgenaue Prüfung, wie bei einem Finanztransfer, wichtig ist. Eine Online-Statusprüfung kann durch Validierungsdienste wie das Online Certificate Status Protocol (OCSP) oder das Server-based Certificate Validation Protocol (SCVP) durchgeführt werden.
\item[Verzeichnungsdienst (Directory Service):]\hfill \\
Der Verzeichnisdienst stellt ein oder mehrere Verzeichnisse bereit, welches alle in der PKI ausgestellten Zertifikate und die dazugehörigen öffentlichen Schlüssel enthält.\footnote{vgl. Rouse, Margaret B (2006)}
\item[Validierungsdienst (Validation Authority, VA):]\hfill \\
Dieser Dienst ermöglicht es, Zertifikate Online in Echtzeit zu Überprüfen (Online Certificate Status Protocol (OCSP) oder Server-based Certificate Validation Protocol (SCVP)).
\item[Dokumentationen:]\hfill \\
Eine PKI beinhaltet in der Regel mehrere Dokumente, welche die Prozesse, Verfahren und Regeln der PKI beschreiben. Diese machen Prozesse, wie den Registrierungsprozess, den Umgang mit den Schlüsseln, die Schlüsselerzeugung, Schutzmechanismen und eventuelle rechtliche Bedingungen für die Teilnehmer der PKI nachvollziehbar.\\
Zu den Dokumenten gehören üblicherweise:
	\begin{itemize}
	\item Certification Policy (CP): beschreibt die selbst PKI und ihr Anforderungsprofil.
	\item Certification Practise Statement (CPS): beschreibt die konkrete Umsetzung der Anforderungen an die PKI die in der CP formuliert wurden.
	\item Policy Disclosure Statement (PDS): Falls das CPS der PKI nicht vollständig veröffentlicht werden soll, bietet das PDS einen Auszug, der veröffentlicht werden kann.
	\end{itemize}
\end{description}
Neben diesen Komponenten machen natürlich auch die Zertifikatseigentümer (Subscriber, z.B. Services, Personen, Server, Router, o.ä.) und Nutzer (Participant), also diejenigen, die den Zertifikaten vertrauen, einen wichtigen Teil der PKI aus. Auf dem gegenseitigen Vertrauensverhältnis beruht eine PKI. \footnote{vgl. Wikipedia B (2016)}
%%%%%%%%%%%%%%%%%%%%%%%%%%%%%%%%%%%%%%%%%%%%%%%%%%%%%%%%%%%%%%%%%%%%%%%%%%%%%
%%%%%%%%%%%%%%%%%%%%%%%%%%%%%%%%%%%%%%%%%%%%%%%%%%%%%%%%%%%%%%%%%%%%%%%%%%%%%
%%%%%%%%%%%%%%%%%%%%%%%%%%%%%%%%%%%%%%%%%%%%%%%%%%%%%%%%%%%%%%%%%%%%%%%%%%%%%
%%%%%%%%%%%%%%%%%%%%%%%%%%%%%%%%%%%%%%%%%%%%%%%%%%%%%%%%%%%%%%%%%%%%%%%%%%%%%
%%%%%%%%%%%%%%%%%%%%%%%%%%%%%%%%%%%%%%%%%%%%%%%%%%%%%%%%%%%%%%%%%%%%%%%%%%%%%
\subsection{Funktionsweise}
In einer hierarchischen PKI gibt es eine Wurzelzertifizierungsinstanz (Root-CA). Sie ist die oberste Zertifizierungsinstanz, der alle anderen Teilnehmer
vertrauen müssen. Tatsächlich gibt es keine globale Root-CA. Viele Länder und größere Unternehmen richten ihre eigenen PKIs mit einer eigenen Wurzelzertifizierungsinstanz ein. Dieses Verhalten entspringt weniger aus mangelnden Vertrauen untereinander, als mehr aus dem Wunsch, die vollkommene Kontrolle über die Vorgänge in der PKI zu haben.\\
\\
Die Integrität der Wurzelinstanz und dessen privater Schlüssel sind essentiell für die sichere Kommunikation innerhalb der PKI, weshalb der Schutz der Root-CA die höchste Priorität hat. Diese sollte daher physisch durch Zugangskontrollen und im Netzwerk durch einen reinen Offline-Betrieb geschützt werden, um einen Zugriff von außen zu verhindern. In diesem Zusammenhang ist auch ein sicheres Verfahren zur Wiederherstellung der Wurzel-Zertifikate bei einem Schaden notwendig.\\
\\
Für die Verarbeitung von Signatur- und Verschlüsselungsanforderungen werden Unterzertifikate genutzt, die durch ein Wurzelzertifikat signiert wurden. Ihr Gültigkeitsdauer ist so kurz gewählt, dass es mit den heutigen technischen Möglichkeiten nicht machbar ist, den privaten Schlüssel der Instanz in diesen Zeitraum zu knacken.\\
\\
Eine Möglichkeit, eine gemeinsame, übergeordnete Instanz für mehrere PKIs zu implementieren, ist die Cross-Zertifizierung. Dabei stellen sich die Wurzelinstanzen der jeweiligen PKIs gegenseitig ein (Cross-)Zertifikat aus. Dabei sind die Regelungen der einen PKI für die andere PKI nicht zwingend anzuwenden, weshalb die Definition der Vertrauensbeziehung zwischen den Parteien schwer festzulegen ist.\\
\\
Das Problem bei dieser bilateralen Zertifizierung ist die Anzahl an Cross-Zertifikaten, die dabei entsteht. Diese steigt quadratisch zur Anzahl der eingebundenen Wurzelzertifizierungsstellen. So entstehen bei 20 Wurzelinstanzen insgesamt 380 Cross-Zertifikate (20 $\cdot$ 19). Die Lösung dieses Problems stellt eine neutrale Bridge-Zertifizierungsstelle dar. Diese tauscht mit allen beteiligten Root-CAs Cross-Zertifikate aus. So lassen sich die Zertifikate jeder beteiligten PKI über die Zertifikate der Bridge-CA zu den Zertifikaten der anderen PKI zurückführen.\\
\\
Die Implementierung einer eigenen PKI ist in der Regel aufwändig und lohnt sich vor allem für größere Unternehmen oder größere Behörden. Kleinere Organisationen verzichten oftmals auf den Aufbau einer solchen Infrastruktur und beziehen ihre Zertifikate von speziellen Dienstleistern.\footnote{vgl. Wikipedia B (2016)}
%%%%%%%%%%%%%%%%%%%%%%%%%%%%%%%%%%%%%%%%%%%%%%%%%%%%%%%%%%%%%%%%%%%%%%%%%%%%%
%%%%%%%%%%%%%%%%%%%%%%%%%%%%%%%%%%%%%%%%%%%%%%%%%%%%%%%%%%%%%%%%%%%%%%%%%%%%%
%%%%%%%%%%%%%%%%%%%%%%%%%%%%%%%%%%%%%%%%%%%%%%%%%%%%%%%%%%%%%%%%%%%%%%%%%%%%%
%%%%%%%%%%%%%%%%%%%%%%%%%%%%%%%%%%%%%%%%%%%%%%%%%%%%%%%%%%%%%%%%%%%%%%%%%%%%%
%%%%%%%%%%%%%%%%%%%%%%%%%%%%%%%%%%%%%%%%%%%%%%%%%%%%%%%%%%%%%%%%%%%%%%%%%%%%%
\subsection{Vertrauensmodelle}
\label{sec:Vertrauensmodelle}
Neben der hierarchischen PKI gibt es noch andere Vertrauensmodell nach denen Zertifikate erstellt und vergeben werden können.\\
\\
\textbf{Direct Trust}\\
Dieses Modell sieht vor, dass sich die Nutzer die Authentizität ihrer Schlüssel gegenseitig bestätigen: Person A bestätig Person B, dass der öffentliche Schlüssel ihm gehört. Person B vertraut Person A. Auch Person C nutzt nun den öffentlichen Schlüssel von Person A, da die Person C der Aussage von Person B vertraut und diese den Schlüssel von A bestätigt.\\
\\
\textbf{Web Of Trust}\\
Beim Web Of Trust geht das Direct Trust Prinzip einen Schritt weiter. In diesem Fall stellen sich die Nutzer gegenseitig Zertifikate aus. Hier ein Beispiel zur Verdeutlichung: Person A möchte eine Nachricht an Person Z mit Hilfe dessen öffentlichen Schlüssels verschicken. Person A weiß jedoch nicht, ob sie den öffentlichen Schlüssel von Person Z trauen kann. Person B hat jedoch, beispielweise über Direct Trust verifiziert, dass der öffentliche Schlüssel tatsächlich Person Z gehört. Daher signiert er diesen Schlüssel. Person A vertraut Person B und weiß aufgrund seiner Signatur, dass der öffentliche Schlüssel von Z \glqq sicher\grqq~ist. Funktioniert die Kommunikation zwischen A und Z erfolgreich, signiert auch Person A den Schlüssel von Z. So entsteht ein Pfad von \glqq vertrauenswürdigen Schlüsseln\grqq~die den unbekannten Schlüssel verifizieren.\\
\\
Für viele Kommunikationsarten ist dieses Prinzip der gegenseitigen Bestätigung jedoch zu unsicher, weshalb in diesen Fällen eine hierarchische PKI bevorzugt wird.\footnote{vgl. Schmeh, Klaus (2001), S.282 ff.}
%%%%%%%%%%%%%%%%%%%%%%%%%%%%%%%%%%%%%%%%%%%%%%%%%%%%%%%%%%%%%%%%%%%%%%%%%%%%%
%%%%%%%%%%%%%%%%%%%%%%%%%%%%%%%%%%%%%%%%%%%%%%%%%%%%%%%%%%%%%%%%%%%%%%%%%%%%%
%%%%%%%%%%%%%%%%%%%%%%%%%%%%%%%%%%%%%%%%%%%%%%%%%%%%%%%%%%%%%%%%%%%%%%%%%%%%%
%%%%%%%%%%%%%%%%%%%%%%%%%%%%%%%%%%%%%%%%%%%%%%%%%%%%%%%%%%%%%%%%%%%%%%%%%%%%%
%%%%%%%%%%%%%%%%%%%%%%%%%%%%%%%%%%%%%%%%%%%%%%%%%%%%%%%%%%%%%%%%%%%%%%%%%%%%%
\section{Fazit}
Zertifikate sind wichtig, um eine sichere Kommunikation mit der Möglichkeit, sich seinem Gegenüber zu versichern, zu ermöglichen. Die Public-Key-Infrastructure bietet die nötige Infrastruktur, um diese Überprüfung zu organisieren und Missbrauch so schnell wie möglich zu unterbinden.\\
\\
Das sensibelste Element dieser Struktur ist die Zertifizierungsstelle, deren Schutz die höchste Priorität haben sollte. Ein erfolgreicher Hackerangriff auf eine Zertifizierungsstelle würde zur Folge haben, dass die Hacker sich selbst seriöse Zertifikate ausstellen können. Selbst wenn die betroffenen CA-Zertifikate gesperrt werden, wird auch den legitimen Zertifikate nicht mehr vertraut und sie werden nicht mehr anerkannt. Je nach größe der PKI kann dies weitreichende Folgen für die IT-Infrastruktur haben, beispielsweise wenn die Zertifikate auch in staatlichen PKIs verwendet werden.\footnote{vgl. Wikipedia B (2016)}\\
\\
Vorfälle aus den vergangenen Jahren haben gezeigt, dass diese Angriffe nicht unrealistisch sind. 2015 musste Lenovo feststellen, dass sich auf vielen ihrer Notebooks eine vorinstallierte Adware als Wurzelzertifizierung im Speicher des Geräts festgesetzt hatte, mit der sich Man-in-the-Middle-Angriffe realisieren ließen.
\footnote{vgl. Bocek, Kevin (2016)}

Der Schutz der eigenen PKI ist aktueller denn je. Aufgrund der technischen Möglichkeiten der Hacker ist ein Angriff nicht immer zu verhindern.
Daher sollte der Fokus darauf liegen, wie man auf Angriffe best- und schnellstmöglich reagiert. Das Risikomanagement spielt hier die wichtigste Rolle.

\author{}
\chapter{Ehrenwörtliche Erklärung}

Hiermit versichern wir diese Arbeit selbstständig verfasst und keine
anderen als die angegebenen Quellen benutzt zu haben.  Wörtliche und
sinngemäße Zitate sind kenntlich gemacht.  Über Zitierrichtlinien
sind wir schriftlich informiert worden.

\renewcommand{\bibname}{Quellenverzeichnis}
\begin{thebibliography}{9}
  \BreakBibliography{\minisec{Monographien}}
\bibitem{KryptoIX} Schmeh, Klaus (2009), \emph{Kryptografie}, 4. Auflage - iX Edition, 2009
\bibitem{Kryptobook} Schmeh, Klaus (2001), \emph{Kryptografie und Public-Key-Infrastrukturen im Internet}, 2. Auflage, Heidelberg, 2001
\bibitem{IT-Sicherheit} Eckert, Claudia (2013), \emph{IT-Sicherheit},
  8.~Auflage, Oldenbourg Verlag, 2013
\BreakBibliography{\minisec{Fachaufsätze}}
\bibitem{SpringerKey} Jens-Matthias Bohli, Christoph Sorge (2008), \emph{Key-Substitution-Angriffe und das Signaturgesetz}, Datenschutz und Datensicherheit - DuD - Volume 32, Issue 6 , pp 388-392, Online im Internet: url{http://link.springer.com/article/10.1007/s11623-008-0093-9\#/page-1}, Stand 5.3.2016
  \BreakBibliography{\minisec{Internetquellen}}
\bibitem{MotleyFool} The Motley Fool (2016), \emph{Paypal Aktie: 5 Schlüssel-Erkenntnisse aus dem Quartalsbericht}, Online im
  Internet: \url{https://www.fool.de/2016/02/08/paypal-aktie-5-schlussel-erkenntnisse-aus-dem-quartalsbericht/}, Stand 17.02.2016
\bibitem{RSAWiki} o.V. (2016), \emph{RSA-Kryptosystem}, Online im
  Internet: \url{https://de.wikipedia.org/wiki/RSA-Kryptosystem}, Stand 17.02.2016
\bibitem{RSAAlgorithm} Evengy Milanov (2009), \emph{The RSA Algorithm}, Online im
  Internet: \url{https://www.math.washington.edu/~morrow/336_09/papers/Yevgeny.pdf}, Stand 19.02.2016
\bibitem{bar} Kleinjung, Aoki, Lenstra, Thom, Bos, Gaudrey, Kruppa, Montgomery, Osvik, Riele,Timofeev, Zimmermann (2010), \emph{Factorization of a 768-bit RSA modulus}, Online im
  Internet: \url{http://eprint.iacr.org/2010/006.pdf}, Stand 18.02.2016
\bibitem{bar} Adi Shamir (o.J.), \emph{Factoring Large Numbers with the TWINKLE Device}, Online im
  Internet: \url{http://citeseerx.ist.psu.edu/viewdoc/download?doi=10.1.1.96.5855&rep=rep1&type=pdf}, Stand 18.02.2016
\bibitem{restklassenring} Rehn, Christian (2012), \emph{Restklassenringe}, Online im
  Internet: \url{http://www.christian-rehn.de/2012/01/17/restklassenringe/}, Stand 09.03.2016
\bibitem{einwegfunktion} o.V. (2001), \emph{Einwegfunktionen}, Online im Internet: \url{http://www.matheprisma.de/Module/RSA/index.htm?5}, Stand 09.03.2016
\bibitem{faktorisierung} Wikipedia, die freie Enzyklopädie (2015), \emph{Faktorisierungsverfahren}, Online im Internet: \url{https://de.wikipedia.org/wiki/Faktorisierungsverfahren}, Stand 09.03.2016
\bibitem{faktorisierung_fermat} Wikipedia, die freie Enzyklopädie (2015), \emph{Faktorisierungsmethode von Fermat}, Online im Internet: \url{https://de.wikipedia.org/wiki/Faktorisierungsverfahren}, Stand 09.03.2016
\bibitem{euler_phi} Steinfeld, Thomas (2015), \emph{Eulersche Phi-Funktion}, Online im Internet: \url{http://www.mathepedia.de/Eulersche_Phi-Funktion.aspx}, Stand 09.03.2016
\bibitem{euklid} Serlo (o.J), \emph{Euklidischer Algorithmus}, Online im Internet: \url{https://de.serlo.org/mathe/zahlen-groessen/teiler-primzahlen/teiler-vielfache/euklidischer-algorithmus}, Stand 09.03.2016
\bibitem{bezout} Wikipedia, die freie Enzyklopädie (2015), \emph{Lemma von Bézout}, Online im Internet: \url{https://de.wikipedia.org/wiki/Lemma_von_Bézout}, Stand 09.03.2016
\bibitem{Search-Article} Bocek, Kevin (2016), \emph{Schlüssel und Zertifikate: Was wir aus 2015 für 2016 lernen sollten}, Online im Internet: \url{http://www.searchsecurity.de/meinung/Schluessel-und-Zertifikate-Was-wir-aus-2015-fuer-2016-lernen-sollten}, Stand 09.03.2016
\bibitem{Freund} Freund, Rosa (2004), \emph{Kryptographie - eine mathematische Einfuhrung}, Online im Internet: \url{http://mirror.eu.oneandone.net/projects/media.ccc.de/congress/2004/papers/214\%20Kryptographie\%20in\%20Theorie\%20und\%20Praxis.pdf}, Stand 09.03.2016
\bibitem{Trier-PKI} Hochschule Trier A (2007), \emph{Public Key Infrastruktur (PKI)}, Online im Internet: \url{http://www.hochschule-trier.de/index.php?id=385}, Stand 09.03.2016
\bibitem{Trier-Z} Hochschule Trier B (2009), \emph{Wozu braucht man ein digitales Zertifikat?}, Online im Internet: \url{http://www.hochschule-trier.de/index.php?id=5850}, Stand 09.03.2016
\bibitem{IT-Wissen} IT-Wissen (2016), \emph{Digitales Zertifikat}, Online im Internet: \url{http://www.itwissen.info/definition/lexikon/Digitales-Zertifikat-digital-certificate.html}, Stand 09.03.2016
\bibitem{Search-DZ} Rouse, Margaret A (2015), \emph{Digitales Zertifikat}, Online im Internet: \url{http://www.searchsecurity.de/definition/Digitales-Zertifikat}, Stand 09.03.2016
\bibitem{Search-PKI} Rouse, Margaret B (2006), \emph{PKI (Public-Key-Infrastruktur)}, Online im Internet: \url{http://www.searchsecurity.de/definition/PKI-Public-Key-Infrastruktur}, Stand 09.03.2016
\bibitem{Wiki-DZ} Die freie Enzyklopädie, Wikipedia A (2014), \emph{Digitales Zertifikat}, Online im Internet: \url{https://de.wikipedia.org/wiki/Digitales_Zertifikat}, Stand 09.03.2016
\bibitem{Wiki-PKI} Die freie Enzyklopädie, Wikipedia B (2016), \emph{Public-Key-Infrastruktur}, Online im Internet: \url{https://de.wikipedia.org/wiki/Public-Key-Infrastruktur}, Stand 09.03.2016
\bibitem{Wiki-PKZ} Die freie Enzyklopädie, Wikipedia C (2015), \emph{Public-Key-Zertifikat}, Online im Internet: \url{https://de.wikipedia.org/wiki/Public-Key-Zertifikat}, Stand 09.03.2016
\bibitem{java-integer-hashing} GrepCode (2015), \emph{GC: Integer -
    java.lang.integer (.java) - GrepCode Class Source}, Online im
  Internet:
  \url{http://grepcode.com/file/repository.grepcode.com/java/root/jdk/openjdk/6-b14/java/lang/Integer.java},
  Stand 10.03.2016
\bibitem{djb2-idea} Bernstein, Daniel J.~(1990), \emph{Hash??? Not
    quite clear on what this is\ldots}, Online im Internet: \url{https://groups.google.com/forum/\#!msg/comp.lang.c/VByoIO8GySs/2XN9iGTpgmsJ}
\bibitem{djb2-impl} Yigit, Ozan (2003), \emph{Hash Functions}, Online
  im Internet: \url{http://www.cse.yorku.ca/~oz/hash.html}
\bibitem{sha3-release} NIST (2015), \emph{NIST releases SHA-3
    Cryptographic Hash Standard}, Online im Internet: \url{http://www.nist.gov/itl/csd/201508_sha3.cfm}
\bibitem{rfc2104} IETF (1997), \emph{RFC 2104 - HMAC: Keyed-Hashing
    for Message Authentication}, Online im Internet: \url{https://tools.ietf.org/html/rfc2104}
  \BreakBibliography{\minisec{Universitätsskripte}}
\bibitem{inverse} Spiegelberg, Thomas (2011), \emph{Bestimmung inverser Elemente in Restklassenringen mittels erweitertem Euklid'schen Algorithmus}, Online im Internet: \url{http://home.in.tum.de/~spiegelb/pdf/Inverseelemente.pdf}, Stand 09.03.2016
\end{thebibliography}

\end{document}
