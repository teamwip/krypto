\documentclass[a4paper, fontsize=12pt, toc=bibliographynumbered]{scrreprt}
\setcounter{tocdepth}{3}
\setcounter{secnumdepth}{3}
\usepackage[T1]{fontenc}
\usepackage[utf8]{inputenc}
\usepackage{mathptmx}
\usepackage{tgheros}
\usepackage{microtype}
\usepackage[onehalfspacing]{setspace}
\usepackage[defaultlines=2, all]{nowidow}
\usepackage[left=4cm, right=2cm, top=2.5cm, bottom=2.5cm]{geometry}
\usepackage[ngerman]{babel}
\usepackage[usenames,dvipsnames]{color}
\usepackage{graphicx}
\usepackage{tabu}
\usepackage[footnote]{acronym}
\usepackage[colorlinks=true, linktoc=page, urlcolor=blue, citecolor=Green]{hyperref}
\usepackage[ngerman]{cleveref}
\makeatletter
\g@addto@macro{\maketitle}{\def\author#1{\def\@author{#1}}}
\newcommand*{\extendsectlevel}[1]{%
  \expandafter\newcommand\expandafter*\csname saved@#1\endcsname{}%
  \expandafter\let\csname saved@#1\expandafter\endcsname\csname #1\endcsname
  \expandafter\renewcommand\expandafter*\csname #1\endcsname{%
    \expandafter\let\csname author@#1\endcsname\@author
    \@ifstar
      {\csname star@#1\endcsname}%
      {\@dblarg{\csname opt@#1\endcsname}}%
  }%
  \expandafter\newcommand\expandafter*\csname star@#1\endcsname[1]{%
    \csname saved@#1\endcsname*{##1%
      \expandafter\ifx\csname author@#1\endcsname\@empty\else
        \hfill\linebreak{\normalsize
          \textmd{\textit{\csname author@#1\endcsname}}}%
      \fi
    }%
  }%
  \expandafter\newcommand\expandafter*\csname opt@#1\endcsname[2][]{%
    \csname saved@#1\endcsname[{##1%
      \expandafter\ifx\csname author@#1\endcsname\@empty\else
        \enskip\textmd{\textit{(\csname author@#1\endcsname)}}%
      \fi
    }]{##2%
      \expandafter\ifx\csname author@#1\endcsname\@empty\else
        \hfill\linebreak{\normalsize
          \textmd{\textit{\csname author@#1\endcsname}}}%
      \fi
    }%
  }%
}
\extendsectlevel{chapter}
\extendsectlevel{section}
\extendsectlevel{subsection}
\makeatother
\begin{document}
\newcommand{\headerrule}{\tabucline -}
\newcommand{\abbildung}[2]{\begin{figure}\centering
    \fbox{\includegraphics[width=0.8\textwidth]{#1}}\caption{#2}
    \label{fig:#1}\end{figure}}
\setkomafont{chapterentry}{\bfseries}

\author{}
\begin{titlepage}
  \begin{center}
    \textsc{\large Fachhochschule der Wirtschaft\\FHDW}\\[1em]
    \textsc{\large Bergisch Gladbach}\\[2em]
    \textsc{Schriftliche Ausarbeitung}\\[6em]
    {\LARGE Kryptographie II}\\[25em]
    \begin{tabu} to 0.8\textwidth {X[l] X[r]}
      \emph{Autoren:}\linebreak
      An-Nam \textsc{Pham},\linebreak
      Leonie \textsc{Schiburr},\linebreak
      Patrick \textsc{Künzel},\linebreak
      Vasilij \textsc{Schneidermann}
      &
      \emph{Pr"ufer:}\linebreak
      Ralf \textsc{Schumann}
    \end{tabu}
    \vfill
    \emph{Abgabetermin:}\\
    \today
  \end{center}
\end{titlepage}

\pagenumbering{roman}
\tableofcontents
\listoffigures
\listoftables
\clearpage
\pagenumbering{arabic}
\setcounter{page}{1}

% Akronyme

\newacro{PKI}{Public Key Infrastructure}
\newacro{RSA}{Rivest, Shamir, Adleman}
\newacro{MD5}{Message Digest 5}
\newacro{SHA1}{Secure Hash Algorithm 1}

\author{Vasilij Schneidermann}
\chapter{Einleitung}

Asymmetrische Kryptographie ist super!

\author{Autor: An-Nam Pham}
\chapter{Mathematische Grundlagen}

\author{Autor: Patrick Künzel}
\chapter{\ac{RSA}}

\author{Autor: Vasilij Schneidermann}
\chapter{Hashing, Signaturen}

\author{Autor: Leonie Schiburr}
\chapter{\ac{PKI}, Zertifikate}


\author{}
\chapter{Zusammenfassung}

Asymmetrische Kryptographie ist toll!

\author{}
\chapter{Ehrenw"ortliche Erkl"arung}

Hiermit versichern wir diese Arbeit selbstst"andig verfasst und keine
anderen als die angegebenen Quellen benutzt zu haben.  W"ortliche und
sinngem"a"se Zitate sind kenntlich gemacht.  "Uber Zitierrichtlinien
sind wir schriftlich informiert worden.

\renewcommand{\bibname}{Quellenverzeichnis}
\begin{thebibliography}{9}
  \BreakBibliography{\minisec{Monographien}}
\bibitem{foo} Doe, John (1999), \emph{Random Monography}, erste
  Auflage, New York, 1999
  \BreakBibliography{\minisec{Internetquellen}}
\bibitem{bar} Anonymous (2016), \emph{Random Rant}, Online im
  Internet: \url{http://example.com/}, Stand 25.02.2016
\end{thebibliography}

\end{document}
