% -*- LaTeX-command: pdflatex; TeX-master: "main.tex"; -*-
\author{Autor: Vasilij Schneidermann}
\chapter{Hashing}

Die Abbildung beliebiger Eingabedaten auf Ausgabedaten fester Größe
wird allgemein als \emph{Hashing} bezeichnet.  Das konkrete Verfahren
ist eine \emph{Hashfunktion}, das Ergebnis des Verfahrens ist der
\emph{Hashwert}\footnote{S.~207, Kryptographie}.

Typische Anwendungsfälle für Hashing sind \emph{Hash Tables} und
\emph{Sets} für die schnelle Suche eines Wertes in einem Wörterbuch,
bzw.~einer Menge, aber auch Authentifizierung und die Sicherstellung
von Integrität eines Dokuments in der Kryptographie.  Für letztere
sind \emph{kryptographische Hashfunktionen} notwendig.

\section{Hashfunktionen}

Die in der Einleitung verwendete Beschreibung kann unter der Annahme,
dass $H$ die Hashfunktion, $h$ der Hashwert und $m$ die zu hashende
Nachricht ist folgendermaßen ausgedrückt werden:

$$h = H(m)$$

Anhand der Anforderung, dass der Wertebereich für $m$ unendlich groß
ist, aber der Wertebereich für $h$ eine endliche Größe hat, kann
geschlussfolgert werden, dass $H$ \emph{nicht} injektiv sein kann.
Eine andere fundamentale Feststellung ist, dass durch diese Abbildung
nicht jede Nachricht einen einzigartigen Hashwert haben kann.
Vorkommnisse verschiedener Nachrichten mit dem gleichen Hashwert
werden als \emph{Kollisionen} bezeichnet\footnote{S.~382,
  IT-Sicherheit}.

Kollisionen sind als solches bei nicht-kryptographischen
Anwendungsfällen eher ein Ärgernis welches z.B.~bei einer Hash Table
gesondert behandelt wird und zu leicht verminderter Performance führt,
bei kryptographischen Anwendungsfällen hingegen werden Verfahren
bevorzugt welche das Risiko von auftretenden Kollisionen
minimieren\footnote{S.~382, IT-Sicherheit}.

Zur Vermeidung von Kollisionen werden üblicherweise zwei weitere
Anforderungen an die Hashfunktion hinzugezogen.  Einerseits sollen
möglichst alle möglichen Hashwerte erreichbar sein oder in anderen
Worten, die errechneten Hashwerte gleichverteilt sein, andererseits
ist es wünschenswert dass selbst minimale Änderungen an der Nachricht
zu vollkommen verschiedenen Hashwerten führen\footnote{S.~208,
  Kryptographie}.

\section{Beispiele nicht-kryptographischer Hashfunktionen}

Falls die Hashfunktion auf einen relativ kleinen Bereich von
Nachrichten angewandt wird, ist es möglich den Wert der Nachricht
selbst als Hashwert zu nutzen.  Dies wird zum Beispiel für das Hashing
vom Datentyp \emph{Integer} im OpenJDK genutzt\footnote{GrepCode
  (2015), Online im Internet}.

Für das Hashing von Strings verwendet man etwas komplexere
Algorithmen.  Ein solches Beispiel wurde von Daniel J.~Bernstein
veröffentlicht\footnote{Bernstein, Daniel J.~(1990), Online im
  Internet}, dieses kann mit wenig Aufwand in C implementiert
werden\footnote{Yigit, Ozan (2003), Online im Internet}:

\begin{lstlisting}[language=C,basicstyle=\ttfamily\footnotesize,keywordstyle=\bfseries\color{black},captionpos=b,caption={djb2}]
unsigned long
hash(unsigned char *str)
{
    unsigned long hash = 5381;
    int c;

    while (c = *str++)
        hash = ((hash << 5) + hash) + c; /* hash * 33 + c */

    return hash;
}
\end{lstlisting}

\section{Kryptographische Hashfunktionen}

Für die Nutzung von Hashfunktionen in der Kryptographie kommen noch
weitere Anforderungen hinzu.  Da Kollisionen bekanntermaßen
existieren, sollte es für einen Angreifer nicht praktikabel sein eine
Kollision zu finden\footnote{S.~209, Kryptographie}.  Dabei wird
zwischen \emph{schwacher Kollisionsresistenz} und \emph{starker
  Kollisionsresistenz} unterschieden.

\subsection{Schwach kollisionsresistente Hashfunktionen}

Gegeben sei die bisherige Definition einer Hashfunktion.  Es müssen
zusätzlich folgende Kriterien erfüllt werden\footnote{S.~382"=383,
  IT-Sicherheit}:

\begin{enumerate}
\item $H$ ist eine Einwegfunktion, d.h.~es ist leicht $h = H(m)$ zu
  berechnen, aber schwer $m = H^{-1}(h)$ zu berechnen
\item Ist $h = H(m)$ gegeben, ist es schwierig eine andere Nachricht
  $m'$ zu finden für welche $h = H(m')$ gilt.
\end{enumerate}

Ist diese Eigenschaft nicht gegeben, ist es für einen Angreifer
möglich eine Nachricht zu seinem Gunsten zu fälschen welche den
gleichen Hashwert aufweist.

\subsection{Stark kollisionsresistente Hashfunktionen}

Folgende Kriterien müssen erfüllt werden\footnote{S.~385}:

\begin{enumerate}
\item $H$ muss eine schwach kollisionsresistente Hashfunktion sein
\item Es ist schwierig $m$ und $m'$ zu finden für welche $H(m) =
  H(m')$ gilt.
\end{enumerate}

\section{Beispiele kryptographischer Hashfunktionen}

\subsection{\ac{MD5}}

MD5 ist aus historischen Gründen eine noch weitverbreitete
kryptographische Hashfunktion, sollte aber aufgrund von
Sicherheitsbedenken nicht mehr eingesetzt werden, da es möglich ist in
relativ kurzer Zeit Kollisionen auf handelsüblicher Hardware zu
finden\footnote{S.~224, Kryptographie}.

\subsection{\ac{SHA1}}

SHA1 ist eine sehr populäre kryptographische Hashfunktion welche unter
anderem in SSL, IPsec und PGP verwendet wird\footnote{S.~221,
  Kryptographie}.  Jedoch wird an der Sicherheit gezweifelt, da es
mehrere Angriffe gegeben hat welche es ermöglichen könnten Kollisionen
herbeizuführen.

\subsection{\ac{SHA2}}

SHA2 ist eine Familie an SHA1-Varianten deren Sicherheit noch unklar
bleibt.  Aufgrunddessen werden sie in der Praxis selten
eingesetzt\footnote{S.~222-223, Kryptographie}.

\subsection{\ac{SHA3}}

SHA3 ist der neueste Standard der SHA-Familie welcher die Schwächen
der bisherigen Varianten vermeidet.  In einem internationalen
Wettbewerb wurde der Keccak-Algorithmus Mitte 2015 auserwählt und ist
damit die Empfehlung für sicheres Hashing\footnote{NIST (2015), Online
  im Internet}.

\section{Bekannte Attacken}

\subsection{Geburtstagsattacke}

Das Geburtstagsparadoxon beschreibt das Phänomen, dass es schon in
einer relativ kleinen Gruppe von Personen mit hoher Wahrscheinlichkeit
zwei mit dem gleichen Geburtstag geben wird.

Ist $k$ die Anzahl an möglichen Geburtstage, kann man die Anzahl $n$
an mindestens benötigten Personen für eine Wahrscheinlichkeit größer
$0.5$ durch $\sqrt{k}$ annähern\footnote{S.~213, Kryptographie}.

Die Formel kann auf Hashwerte ebenfalls angewendet werden.  Wenn
dieser eine feste Länge $x$ hat, beträgt $k$ dann $2^x$.  $n$ wäre in
diesem Fall also $2^{\frac{x}{2}}$, d.h.~durch das
Geburtstagsparadoxon wird die Schlüssellänge effektiv halbiert.  Ein
Angreifer müsste also in der Lage sein sowohl diese Anzahl an
Hashwerten errechnen sowie speichern zu können.  Deswegen verwendet
man in der Praxis mehr als die doppelte Schlüssellänge von dem was ein
Angreifer bewältigt bekommen würde, im Falle von SHA1 z.B.~160 Bits.

\subsection{Substitutionsattacke}

Kennt der Angreifer den Hashwert einer Nachricht die er zu seinem
Gunsten fälschen möchte, kann er eine andere Nachricht konstruieren,
Stellen finden welche er gefahrlos ersetzen kann ohne das Aussehen
oder die Bedeutung der Nachricht zu ändern und alle Kombinationen
möglicher Änderungen hashen.  Bei $n$ Änderungen wären dies $2^n$
denkbare Kombinationen.  Ist die Anzahl an Kombinationen hinreichend
hoch, ist es möglich eine gefälschte Nachricht mit dem gleichen
Hashwert wie die originale Nachricht zu finden.  Die
Wahrscheinlichkeit dafür ist besonders hoch wenn die Schlüssellänge
$x$ des Hashwertes kleiner $n$ ist, d.h.~es ist ratsam Hashwerte mit
hoher Schlüssellänge zu nutzen\footnote{S.~211"=212, Kryptographie}.

\subsection{Wörterbuchattacke}

Wird eine kurze Nachricht gehasht (wie z.B.~ein Passwort), ist eine
Wörterbuchattacke möglich.  Anstatt alle denkbaren Nachrichten
auszuprobieren, verwendet man ein Wörterbuch und probiert seine
Einträge durch.  Dieser Angriff ist erstaunlich effektiv in der
Praxis, da eine Menge Benutzer nicht auf sichere Passwörter setzen.

Eine mögliche Maßnahme zur Erhöhung des Angreiferaufwands ist das
\emph{Salt}, ein zusammen mit jedem einzelnen Hashwert gespeicherter
Wert welcher zusammen mit der Nachricht gehasht wurde.  Während diese
Vorgehensweise es für den Angreifer nicht schwieriger macht einen
einzelnen Hashwert zu knacken, führt sie dazu, dass er nicht mehrere
Hashwerte parallel knacken kann, da für jeden ein anderes Salt
verwendet wird\footnote{S.~214, Kryptographie}.

\emph{Key Stretching} ist eine andere solche Maßnahme.  Anstatt den
Hashwert an sich zu nehmen, wird er mit einer hinreichend sicheren
Anzahl an Vorgängen wiederholt bearbeitet.  Die einfachste Variante
davon ist das Resultat der Hashfunktion erneut als Eingabe zu nutzen.
Dadurch werden Angreifer gezwungen für jeden Versuch die gleiche
Anzahl an Hash-Iterationen zu verwenden\footnote{S.~215,
  Kryptographie}.

\subsection{Rainbow Tables}

Legt der Angreifer eine persistente Tabelle von sortierten Nachrichten
und ihren Hashwerten an, kann er diese verwenden um bekannte Hashwerte
schnell nachzuschlagen.  Auf diese Weise werden Bruteforce-Attacken
auf kurze Passwörter gangbar, eine Kombination mit Wörterbuchattacken
ist ebenfalls denkbar.  Salts und Key Stretching sind wirksame
Abhilfen, da man durch sie größere Rainbow Tables erfordert welche
womöglich nicht mehr praktikabel für den Angreifer
sind\footnote{S.~215"=217, Kryptographie}.

\author{Autor: Vasilij Schneidermann}
\chapter{Signaturen}

Mit den bisher vorgestellten Verfahren war es möglich die Integrität
einer Nachricht zu beweisen, jedoch nicht die Authentizität der
Kommunikationspartner.  Für diese zusätzliche Anforderung werden
\ac{MAC}s verwendet\footnote{S.~393, IT-Sicherheit}.  Benötigt man
noch Verbindlichkeit um das Abstreiten der Urheberschaft
auszuschließen, ist eine digitale Signatur notwendig\footnote{S.~240,
  Kryptographie}.

\section{\ac{MAC}s}

Ein \ac{MAC} ist nichts anderes als eine parametrisierte Hashfunktion,
d.h.~eine Hashfunktion welche zusätzlich zu der Nachricht $m$ noch
einen Schlüssel $k$ verwendet.  Die Kommunikationspartner einigen sich
auf einen Schlüssel und verwenden diesen um einen \ac{MAC} der
gesendeten und empfangenen Nachricht zu errechnen.  Stimmt dieser
überein, ist die Authentizität des Dokuments
gewährleistet\footnote{S.~394, IT-Sicherheit}.

\section{Signaturen mit \ac{HMAC}s}

Bei dem \ac{HMAC}-Verfahren\footnote{IETF (1997), Online im Internet}
wird eine bekannte kryptographische Hashfunktion für diesen
Einsatzzweck verwendet.  In der Praxis kommen dafür MD5 und SHA1 in
Frage.  Sind zusätzlich zum Schlüssel $k$ und Nachricht $m$ noch
inneres und äußeres \emph{Padding} \texttt{ipad} und \texttt{opad}
gegeben, errechnet sich der \ac{HMAC} wie folgt\footnote{S.~396"=397,
  IT-Sicherheit}:

$$HMAC(k, m) = H(K \oplus \texttt{opad} \parallel H(K \oplus \texttt{ipad} \parallel m))$$

$\oplus$ symbolisiert hierbei den XOR-Operator, $\parallel$ ist die
String-Konkatenation der Resultate.  \texttt{ipad} ist das Byte
\texttt{0x36} und \texttt{opad} das Byte \texttt{0x5C}, beide
wiederholt bis die Eingaben auf die für die Hashfunktion erforderliche
Blocklänge aufgefüllt sind.

\section{Digitale Signaturen}

Eine digitale Signatur ist das rechtssichere digitale Äquivalent zur
handschriftlichen Signatur.  Sie muss die folgenden Anforderungen
erfüllen\footnote{S.~184, Kryptographie}:

\begin{itemize}
\item Identifikation des Urhebers
\item Echtheit des Dokuments überprüfbar
\item Fälschbarkeit des Dokuments nicht möglich
\item Integrität des Dokuments bleibt erhalten
\end{itemize}

Aufgrund des zweiten Kriteriums können \ac{MAC}s nicht verwendet
werden.  Die Echtheit des Dokuments muss nämlich von einer Drittperson
bestätigt werden welche den gemeinsamen Schlüssel der
Kommunikationspartner \emph{nicht} kennt.

Glücklicherweise lässt sich für diesen Anwendungszweck das
\ac{RSA}-Verfahren nutzen\footnote{S.~185, Kryptographie}.  Anstatt
eine Nachricht mit einem privaten Schlüssel zu verschlüsseln und einem
öffentlichen Schlüssel zu entschlüsseln, wird die Nachricht mit dem
privaten Schlüssel entschlüsselt.  Das Resultat ist die digitale
Signatur.  Verschlüsselt man diese mit dem öffentlichen Schlüssel,
ergibt sich die originale Nachricht und der Beweis für die Echtheit
der digitalen Signatur ist erbracht.
