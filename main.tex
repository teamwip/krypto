\documentclass[a4paper, fontsize=12pt, parskip=full, toc=bibliographynumbered]{scrreprt}
\setcounter{tocdepth}{3}
\setcounter{secnumdepth}{3}
\usepackage[T1]{fontenc}
\usepackage[utf8]{inputenc}
\usepackage{mathptmx}
\usepackage{tgheros}
\usepackage{courier}
\usepackage{microtype}
\usepackage[onehalfspacing]{setspace}
\usepackage[defaultlines=2, all]{nowidow}
\usepackage[left=4cm, right=2cm, top=2.5cm, bottom=2.5cm]{geometry}
\usepackage[ngerman]{babel}
\usepackage[usenames,dvipsnames]{color}
\usepackage{graphicx}
\usepackage{tabu}
\usepackage{listings}
\usepackage{float}
\usepackage[footnote]{acronym}
\usepackage[ngerman]{cleveref}
\usepackage{eurosym}
\usepackage{amsmath}
\usepackage{amssymb}
\usepackage{thmbox}
\usepackage{shadethm}
\usepackage{booktabs}
\usepackage{hyperref}
\makeatletter
\g@addto@macro{\maketitle}{\def\author#1{\def\@author{#1}}}
\newcommand*{\extendsectlevel}[1]{%
  \expandafter\newcommand\expandafter*\csname saved@#1\endcsname{}%
  \expandafter\let\csname saved@#1\expandafter\endcsname\csname #1\endcsname
  \expandafter\renewcommand\expandafter*\csname #1\endcsname{%
    \expandafter\let\csname author@#1\endcsname\@author
    \@ifstar
      {\csname star@#1\endcsname}%
      {\@dblarg{\csname opt@#1\endcsname}}%
  }%
  \expandafter\newcommand\expandafter*\csname star@#1\endcsname[1]{%
    \csname saved@#1\endcsname*{##1%
      \expandafter\ifx\csname author@#1\endcsname\@empty\else
        \hfill\linebreak{\normalsize
          \textmd{\textit{\csname author@#1\endcsname}}}%
      \fi
    }%
  }%
  \expandafter\newcommand\expandafter*\csname opt@#1\endcsname[2][]{%
    \csname saved@#1\endcsname[{##1%
      \expandafter\ifx\csname author@#1\endcsname\@empty\else
        \enskip\textmd{\textit{(\csname author@#1\endcsname)}}%
      \fi
    }]{##2%
      \expandafter\ifx\csname author@#1\endcsname\@empty\else
        \hfill\linebreak{\normalsize
          \textmd{\textit{\csname author@#1\endcsname}}}%
      \fi
    }%
  }%
}
\extendsectlevel{chapter}
\makeatother
\begin{document}
\newcommand{\headerrule}{\tabucline -}
\newcommand{\abbildung}[2]{\begin{figure}\centering
    \fbox{\includegraphics[width=0.8\textwidth]{#1}}\caption{#2}
    \label{fig:#1}\end{figure}}
\setkomafont{chapterentry}{\bfseries}

\author{}
\begin{titlepage}
  \begin{center}
    \textsc{\large Fachhochschule der Wirtschaft\\FHDW}\\[1em]
    \textsc{\large Bergisch Gladbach}\\[2em]
    \textsc{Schriftliche Ausarbeitung}\\[6em]
    {\LARGE Kryptographie II}\\[25em]
    \begin{tabu} to 0.8\textwidth {X[l] X[r]}
      \emph{Autoren:}\linebreak
      An-Nam \textsc{Pham},\linebreak
      Leonie \textsc{Schiburr},\linebreak
      Patrick \textsc{Künzl},\linebreak
      Vasilij \textsc{Schneidermann}
      &
      \emph{Prüfer:}\linebreak
      Ralf \textsc{Schumann}
    \end{tabu}
    \vfill
    \emph{Abgabetermin:}\\
    \today
  \end{center}
\end{titlepage}

% Akronyme

\newacro{PKI}{Public Key Infrastructure}
\newacro{RSA}{Rivest, Shamir, Adleman}
\newacro{MAC}{Message Authentication Code}
\newacro{HMAC}{Hash-based Message Authentication Code}
\newacro{MD5}{Message Digest 5}
\newacro{SHA1}{Secure Hash Algorithm 1}
\newacro{SHA2}{Secure Hash Algorithm 2}
\newacro{SHA3}{Secure Hash Algorithm 3}

\pagenumbering{roman}
\tableofcontents
\listoffigures
\listoftables
\lstlistoflistings
\clearpage
\pagenumbering{arabic}
\setcounter{page}{1}

\author{Vasilij Schneidermann}
\chapter{Einleitung}

Asymmetrische Kryptographie ist super!

\author{Autor: An-Nam Pham}
\chapter{Mathematische Grundlagen}

\author{Autor: Patrick Künzel}
\chapter{\ac{RSA}}

\author{Autor: Vasilij Schneidermann}
\chapter{Hashing, Signaturen}

\author{Autor: Leonie Schiburr}
\chapter{\ac{PKI}, Zertifikate}


\author{}
\chapter{Ehrenwörtliche Erklärung}

Hiermit versichern wir diese Arbeit selbstständig verfasst und keine
anderen als die angegebenen Quellen benutzt zu haben.  Wörtliche und
sinngemäße Zitate sind kenntlich gemacht.  Über Zitierrichtlinien
sind wir schriftlich informiert worden.

\renewcommand{\bibname}{Quellenverzeichnis}
\begin{thebibliography}{9}
  \BreakBibliography{\minisec{Monographien}}
\bibitem{foo} Doe, John (1999), \emph{Random Monography}, erste
  Auflage, New York, 1999
  %%%%%%%%%%%%%%%%%%%%%%%%%%%%%%%%%%%%%%%%%%%%%%%%%%%%%%%
  %%%%%%%%%%%%%%% RSA - VERSCHLÜSSELUNG %%%%%%%%%%%%%%%%%
\bibitem{KryptoIX} Schmeh, Klaus (2009), \emph{Kryptografie}, 4. Auflage - iX Edition, 2009
\bibitem{Kryptobook} Schmeh, Klaus (2001), \emph{Kryptografie und Public-Key-Infrastrukturen im Internet}, 2. Auflage, Heidelberg, 2001
\BreakBibliography{\minisec{Fachaufsätze}}
\bibitem{SpringerKey} Jens-Matthias Bohli, Christoph Sorge (2008), \emph{Key-Substitution-Angriffe und das Signaturgesetz}, Datenschutz und Datensicherheit - DuD - Volume 32, Issue 6 , pp 388-392, Online im Internet: url{http://link.springer.com/article/10.1007/s11623-008-0093-9\#/page-1}, Stand 5.3.2016
  \BreakBibliography{\minisec{Internetquellen}}
\bibitem{bar} Anonymous (2016), \emph{Random Rant}, Online im
  Internet: \url{http://example.com/}, Stand 25.02.2016
%%%%%%%%%%%%%%%%%%%%%%%%%%%%%%%%%%%%%%%%%%%%%%%%%%%%%%%
%%%%%%%%%%%%%%% RSA - VERSCHLÜSSELUNG %%%%%%%%%%%%%%%%%
\bibitem{MotleyFool} The Motley Fool (2016), \emph{Paypal Aktie: 5 Schlüssel-Erkenntnisse aus dem Quartalsbericht}, Online im
  Internet: \url{https://www.fool.de/2016/02/08/paypal-aktie-5-schlussel-erkenntnisse-aus-dem-quartalsbericht/}, Stand 17.02.2016
\bibitem{RSAWiki} o.V. (2016), \emph{RSA-Kryptosystem}, Online im
  Internet: \url{https://de.wikipedia.org/wiki/RSA-Kryptosystem}, Stand 17.02.2016
\bibitem{RSAAlgorithm} Evengy Milanov (2009), \emph{The RSA Algorithm}, Online im
  Internet: \url{https://www.math.washington.edu/~morrow/336_09/papers/Yevgeny.pdf}, Stand 19.02.2016
\bibitem{bar} Kleinjung, Aoki, Lenstra, Thom, Bos, Gaudrey, Kruppa, Montgomery, Osvik, Riele,Timofeev, Zimmermann (2010), \emph{Factorization of a 768-bit RSA modulus}, Online im
  Internet: \url{http://eprint.iacr.org/2010/006.pdf}, Stand 18.02.2016
\bibitem{bar} Adi Shamir (o.J.), \emph{Factoring Large Numbers with the TWINKLE Device}, Online im
  Internet: \url{http://citeseerx.ist.psu.edu/viewdoc/download?doi=10.1.1.96.5855&rep=rep1&type=pdf}, Stand 18.02.2016
\bibitem[Rehn, 2012]{restklassenring} Rehn, Christian (2012), \emph{Restklassenringe}, Online im
  Internet: \url{http://www.christian-rehn.de/2012/01/17/restklassenringe/}, Stand 09.03.2016
\bibitem[MathePrisma, 2001]{einwegfunktion} o.V. (2001), \emph{Einwegfunktionen}, Online im Internet: \url{http://www.matheprisma.de/Module/RSA/index.htm?5}, Stand 09.03.2016
\bibitem[Wikipedia (a), 2015]{faktorisierung} Wikipedia, die freie Enzyklopädie (2015), \emph{Faktorisierungsverfahren}, Online im Internet: \url{https://de.wikipedia.org/wiki/Faktorisierungsverfahren}, Stand 09.03.2016
\bibitem[Wikipedia (b), 2015]{faktorisierung_fermat} Wikipedia, die freie Enzyklopädie (2015), \emph{Faktorisierungsmethode von Fermat}, Online im Internet: \url{https://de.wikipedia.org/wiki/Faktorisierungsverfahren}, Stand 09.03.2016
\bibitem[Steinfeld, 2015]{euler_phi} Steinfeld, Thomas (2015), \emph{Eulersche Phi-Funktion}, Online im Internet: \url{http://www.mathepedia.de/Eulersche_Phi-Funktion.aspx}, Stand 09.03.2016
\bibitem[Serlo, o.J]{euklid} Serlo (o.J), \emph{Euklidischer Algorithmus}, Online im Internet: \url{https://de.serlo.org/mathe/zahlen-groessen/teiler-primzahlen/teiler-vielfache/euklidischer-algorithmus}, Stand 09.03.2016
\bibitem[Wikipedia (c), 2015]{bezout} Wikipedia, die freie Enzyklopädie (2015), \emph{Lemma von Bézout}, Online im Internet: \url{https://de.wikipedia.org/wiki/Lemma_von_Bézout}, Stand 09.03.2016
\bibitem{Search-Article} Bocek, Kevin (2016), \emph{Schlüssel und Zertifikate: Was wir aus 2015 für 2016 lernen sollten}, Online im Internet: \url{http://www.searchsecurity.de/meinung/Schluessel-und-Zertifikate-Was-wir-aus-2015-fuer-2016-lernen-sollten}, Stand 09.03.2016
\bibitem{Freund} Freund, Rosa (2004), \emph{Kryptographie - eine mathematische Einfuhrung}, Online im Internet: \url{http://mirror.eu.oneandone.net/projects/media.ccc.de/congress/2004/papers/214\%20Kryptographie\%20in\%20Theorie\%20und\%20Praxis.pdf}, Stand 09.03.2016
\bibitem{Trier-PKI} Hochschule Trier A (2007), \emph{Public Key Infrastruktur (PKI)}, Online im Internet: \url{http://www.hochschule-trier.de/index.php?id=385}, Stand 09.03.2016
\bibitem{Trier-Z} Hochschule Trier B (2009), \emph{Wozu braucht man ein digitales Zertifikat?}, Online im Internet: \url{http://www.hochschule-trier.de/index.php?id=5850}, Stand 09.03.2016
\bibitem{IT-Wissen} IT-Wissen (2016), \emph{Digitales Zertifikat}, Online im Internet: \url{http://www.itwissen.info/definition/lexikon/Digitales-Zertifikat-digital-certificate.html}, Stand 09.03.2016
\bibitem{Search-DZ} Rouse, Margaret A (2015), \emph{Digitales Zertifikat}, Online im Internet: \url{http://www.searchsecurity.de/definition/Digitales-Zertifikat}, Stand 09.03.2016
\bibitem{Search-PKI} Rouse, Margaret B (2006), \emph{PKI (Public-Key-Infrastruktur)}, Online im Internet: \url{http://www.searchsecurity.de/definition/PKI-Public-Key-Infrastruktur}, Stand 09.03.2016
\bibitem{Wiki-DZ} Die freie Enzyklopädie, Wikipedia A (2014), \emph{Digitales Zertifikat}, Online im Internet: \url{https://de.wikipedia.org/wiki/Digitales_Zertifikat}, Stand 09.03.2016
\bibitem{Wiki-PKI} Die freie Enzyklopädie, Wikipedia B (2016), \emph{Public-Key-Infrastruktur}, Online im Internet: \url{https://de.wikipedia.org/wiki/Public-Key-Infrastruktur}, Stand 09.03.2016
\bibitem{Wiki-PKZ} Die freie Enzyklopädie, Wikipedia C (2015), \emph{Public-Key-Zertifikat}, Online im Internet: \url{https://de.wikipedia.org/wiki/Public-Key-Zertifikat}, Stand 09.03.2016
  \BreakBibliography{\minisec{Universitätsskripte}}
\bibitem[Spiegelberg, 2011]{inverse} Spiegelberg, Thomas (2011), \emph{Bestimmung inverser Elemente in Restklassenringen mittels erweitertem Euklid'schen Algorithmus}, Online im Internet: \url{http://home.in.tum.de/~spiegelb/pdf/Inverseelemente.pdf}, Stand 09.03.2016
\end{thebibliography}

\end{document}
