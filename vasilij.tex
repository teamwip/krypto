% -*- LaTeX-command: pdflatex; TeX-master: "main.tex"; -*-
\author{Autor: Vasilij Schneidermann}
\chapter{Hashing}

Die Abbildung beliebiger Eingabedaten auf Ausgabedaten fester Größe
wird allgemein als \emph{Hashing} bezeichnet.  Das konkrete Verfahren
ist eine \emph{Hashfunktion}, das Ergebnis des Verfahrens ist der
\emph{Hashwert}\footnote{S.~207, Kryptographie}.

Typische Anwendungsfälle für Hashing sind \emph{Hash Tables} und
\emph{Sets} für die schnelle Suche eines Wertes in einem Wörterbuch,
bzw.~einer Menge, aber auch Authentifizierung und die Sicherstellung
von Integrität eines Dokuments in der Kryptographie.  Für letztere
sind \emph{kryptographische Hashfunktionen} notwendig.

\section{Hashfunktionen}

Die in der Einleitung verwendete Beschreibung kann unter der Annahme,
dass $H$ die Hashfunktion, $h$ der Hashwert und $m$ die zu hashende
Nachricht ist folgendermaßen ausgedrückt werden:

$$h = H(m)$$

Anhand der Anforderung, dass der Wertebereich für $m$ unendlich groß
ist, aber der Wertebereich für $h$ eine endliche Größe hat, kann
geschlussfolgert werden, dass $H$ \emph{nicht} injektiv sein kann.
Eine andere fundamentale Feststellung ist, dass durch diese Abbildung
nicht jede Nachricht einen einzigartigen Hashwert haben kann.
Vorkommnisse verschiedener Nachrichten mit dem gleichen Hashwert
werden als \emph{Kollisionen} bezeichnet\footnote{S.~382,
  IT-Sicherheit}.

Kollisionen sind als solches bei nicht-kryptographischen
Anwendungsfällen eher ein Ärgernis welches z.B.~bei einer Hash Table
gesondert behandelt wird und zu leicht verminderter Performance führt,
bei kryptographischen Anwendungsfällen hingegen werden Verfahren
bevorzugt welche das Risiko von auftretenden Kollisionen
minimieren\footnote{S.~382, IT-Sicherheit}.

Zur Vermeidung von Kollisionen werden üblicherweise zwei weitere
Anforderungen an die Hashfunktion hinzugezogen.  Einerseits sollen
möglichst alle möglichen Hashwerte erreichbar sein oder in anderen
Worten, die errechneten Hashwerte gleichverteilt sein, andererseits
ist es wünschenswert dass selbst minimale Änderungen an der Nachricht
zu vollkommen verschiedenen Hashwerten führen\footnote{S.~208,
  Kryptographie}.

\section{Beispiele nicht-kryptographischer Hashfunktionen}

Falls die Hashfunktion auf einen relativ kleinen Bereich von
Nachrichten angewandt wird, ist es möglich den Wert der Nachricht
selbst als Hashwert zu nutzen.  Dies wird zum Beispiel für das Hashing
vom Datentyp \emph{Integer} im OpenJDK
genutzt\footnote{\url{http://grepcode.com/file/repository.grepcode.com/java/root/jdk/openjdk/6-b14/java/lang/Integer.java\#Integer.hashCode\%28\%29}}.

Für das Hashing von Strings verwendet man etwas komplexere
Algorithmen.  Ein solches Beispiel wurde von Daniel J.~Bernstein
veröffentlicht\footnote{\url{https://groups.google.com/forum/\#!msg/comp.lang.c/VByoIO8GySs/2XN9iGTpgmsJ}},
dieses kann mit wenig Aufwand in C implementiert werden\footnote{\url{http://www.cse.yorku.ca/~oz/hash.html}}:

\begin{lstlisting}[language=C,basicstyle=\ttfamily\footnotesize,keywordstyle=\bfseries\color{black},captionpos=b,caption={djb2}]
unsigned long
hash(unsigned char *str)
{
    unsigned long hash = 5381;
    int c;

    while (c = *str++)
        hash = ((hash << 5) + hash) + c; /* hash * 33 + c */

    return hash;
}
\end{lstlisting}

\section{Kryptographische Hashfunktionen}

Für die Nutzung von Hashfunktionen in der Kryptographie kommen noch
weitere Anforderungen hinzu.  Da Kollisionen bekanntermaßen
existieren, sollte es für einen Angreifer nicht praktikabel sein eine
Kollision zu finden\footnote{S.~209, Kryptographie}.  Dabei wird
zwischen \emph{schwacher Kollisionsresistenz} und \emph{starker
  Kollisionsresistenz} unterschieden.

\subsection{Schwach kollisionsresistente Hashfunktionen}

Gegeben sei die bisherige Definition einer Hashfunktion.  Es müssen
zusätzlich folgende Kriterien erfüllt werden\footnote{S.~382"=383,
  IT-Sicherheit}:

\begin{enumerate}
\item $H$ ist eine Einwegfunktion, d.h.~es ist leicht $h = H(m)$ zu
  berechnen, aber schwer $m = H^{-1}(h)$ zu berechnen
\item Ist $h = H(m)$ gegeben, ist es schwierig eine andere Nachricht
  $m'$ zu finden für welche $h = H(m')$ gilt.
\end{enumerate}

Ist diese Eigenschaft nicht gegeben, ist es für einen Angreifer
möglich eine Nachricht zu seinem Gunsten zu fälschen welche den
gleichen Hashwert aufweist.

\subsection{Stark kollisionsresistente Hashfunktionen}

Folgende Kriterien müssen erfüllt werden\footnote{S.~385}:

\begin{enumerate}
\item $H$ muss eine schwach kollisionsresistente Hashfunktion sein
\item Es ist schwierig $m$ und $m'$ zu finden für welche $H(m) =
  H(m')$ gilt.
\end{enumerate}

\section{Beispiele kryptographischer Hashfunktionen}

\section{Bekannte Attacken}

\author{Autor: Vasilij Schneidermann}
\chapter{Signaturen}

Involvieren schwarze Magie.

\section{\ac{MAC}s}

\section{Signaturen mit \ac{MAC}s}

\section{Alternative Signaturverfahren}
